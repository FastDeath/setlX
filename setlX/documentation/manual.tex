\documentclass[a4paper,oneside, DIV=13]{scrartcl}
\def\myDocumentType{article}

% conditionals
\usepackage{ifthen}

% enable special German/European characters
%\usepackage[official]{eurosym}
\usepackage{ucs}
\usepackage[utf8x]{inputenc}

%% set available languages
%\usepackage[ngerman,english]{babel}

% enable (code) listings
\usepackage{listings}

%% enable graphics
%\usepackage{graphicx}

% add hyperlinks
\usepackage[
%	pdftex, % use driver `close' to pdfLaTeX
	unicode=true, % use unicode strings
	bookmarksopen=true, % expand bookmarks by default
	bookmarksnumbered=true % show chapter numbers in bookmarks
]{hyperref}

% point links to top left corner of respective element, not to caption
\usepackage[all]{hypcap}

% enable bold-face in typewriter
\DeclareFontShape{OT1}{cmtt}{bx}{n}{<5><6><7><8><9><10><10.95><12><14.4><17.28><20.74><24.88>cmttb10}{}

%%prerender some unicode characters, to enable their use in predefined commands like \title etc
%\PrerenderUnicode{ÄäÖöÜüß}

%% same for listings
%\lstset{literate= {Ö}{{\"O}}1 {Ä}{{\"A}}1 {Ü}{{\"U}}1 {ß}{{\ss}}2 {ü}{{\"u}}1 {ä}{{\"a}}1 {ö}{{\"o}}1 }

% booleans for settings
\newboolean{linkHighlighting}
\newboolean{tablesAsFigures}
\newboolean{listingsAsFigures}
\newboolean{tocAllingnNonChaptersWithNumberedChapters}
\newboolean{tocAllingnNonSectionsWithNumberedSections}

% load commands
% new title etc commands, which set latex, pdf attribute and defines \gtitle etc for later use
\newcommand{\globalTitle}[1]{
	\title{#1}
	\hypersetup{pdftitle={#1}}
	\def\gTitle{#1}
}
\newcommand{\globalSubject}[1]{
	\ifdefined\subject
		\subject{#1}
	\fi
	\hypersetup{pdfsubject={#1}}
	\def\gSubject{#1}
}
\newcommand{\globalAuthor}[1]{
	\author{#1}
	\hypersetup{pdfauthor={#1}}
	\def\gAuthor{#1}
}
\newcommand{\globalDate}[1]{
	\date{#1}
	\def\gDate{#1}
}
\newcommand{\globalKeywords}[1]{
	\hypersetup{pdfkeywords={#1}}
	\def\gKeywords{#1}
}

% chapter/section without number, but listed in toc
\newcommand{\nonChapter}[1]{
	\chapter*{#1}
	\ifthenelse{\boolean{tocAllingnNonChaptersWithNumberedChapters}}{
	  % then
		\addcontentsline{toc}{chapter}{\hspace{1.35em}#1}
	}{
	  % else
		\addcontentsline{toc}{chapter}{#1}
	}
}
\newcommand{\nonSection}[1]{
	\section*{#1}
	\ifthenelse{\boolean{tocAllingnNonSectionsWithNumberedSections}}{
	  % then
	  	\ifdefined\chapter
			\addcontentsline{toc}{section}{---\hspace{1.3em}#1}
		\else
			\addcontentsline{toc}{section}{\hspace{1.35em}#1}
		\fi
	}{
	  % else
	  	\ifdefined\chapter
			\addcontentsline{toc}{section}{--- #1}
		\else
			\addcontentsline{toc}{section}{#1}
		\fi
	}
}
\newcommand{\nonSubsection}[1]{
	\subsection*{#1}
	\ifthenelse{\boolean{tocAllingnNonSectionsWithNumberedSections}}{
	  % then
	  	\ifdefined\chapter
			\addcontentsline{toc}{subsection}{---\hspace{2.2em}#1}
		\else
			\addcontentsline{toc}{subsection}{---\hspace{1.3em}#1}
		\fi
	}{
	  % else
		\addcontentsline{toc}{subsection}{--- #1}
	}
}
\newcommand{\nonSubsubsection}[1]{
	\subsubsection*{#1}
	\ifthenelse{\boolean{tocAllingnNonSectionsWithNumberedSections}}{
	  % then
	  	\ifdefined\chapter
	  		% usually subsubsections will not be displayed in toc if chapters are first level
			\addcontentsline{toc}{subsubsection}{---\hspace{3.1em}#1}
		\else
			\addcontentsline{toc}{subsubsection}{---\hspace{2.2em}#1}
		\fi
	}{
	  % else
		\addcontentsline{toc}{subsubsection}{--- #1}
	}
}

% basic figure/table command, which centers, provides a caption and labels
\newcommand{\basicIncludeStructure}[6]{
	\begin{#1}[#2] % position
		\begin{center}
		{#3} % content
		\caption[#5]{#6\label{#4}} % short caption | caption | label
		\end{center}
	\end{#1}
}

% new figure command, which centers, provides a caption and labels
\newcommand{\includeFigure}[5][h]{
	\basicIncludeStructure
		{figure} % type
		{#1} % position
		{#2} % figure content
		{#3} % label
		{#4} % short caption
		{#5} % caption
}

% new table command, which provides a caption and labels
\newcommand{\includeTable}[5][h]{
	\ifthenelse{\boolean{tablesAsFigures}}{
	  % then
		\def\tableIncludeType{figure}
	}{
	  % else
		\def\tableIncludeType{table}
	}
	\basicIncludeStructure
		{\tableIncludeType}
		{#1} % position
		{#2} % table content
		{#3} % label
		{#4} % short caption
		{#5} % caption
}

% new listing command, which provides a caption and labels
\newcommand{\includeListing}[6][h]{
	\ifthenelse{\boolean{listingsAsFigures}}{
	  % then
		\includeFigure
			[#1] % position
			{ \lstinputlisting[#3]{#2} } % options | path/filename
			{#4} % label
			{#5} % short caption
			{#6} % caption
	}{
	  % else
		\begin{singlespace}
			\hspace{1em}
			\lstinputlisting[#3,caption={[#5]{#6}},label={#4}]{#2} % options | short caption | caption | label | path/filename
		\end{singlespace}
	}
}

%% new picture command, which centers, provides a caption and labels
%\newcommand{\includePicture}[6][h]{
%	\includeFigure
%		[#1] % position
%		{ \includegraphics[#3]{#2} } % options | path/filename
%		{#4} % label
%		{#5} % short caption
%		{#6} % caption
%}

% mail address with reduced margin for error
\newcommand{\mailto}[1]{\mbox{\href{mailto:#1}{\texttt{#1}}}} % teletype style makes it look similar to \url{}

% strikes a line through the argument text
\newlength{\strikeOutLen} % to save length of line
\newcommand{\strikeOut}[1]{
	\settowidth{\strikeOutLen}{#1} % set length of line
	\mbox{#1\protect\hspace{-\strikeOutLen}\rule[0.5ex]{\strikeOutLen}{0.1ex}}
}

% full line \framebox for comments etc
\newcommand{\boxed}[1]{
	\par
	\noindent
	\framebox[\linewidth]{
		\addtolength{\linewidth}{-5mm}
		\parbox{\linewidth}{#1}
		\addtolength{\linewidth}{5mm}
	}
	\par
}

% universal box used for warnings, todos, etc
\newcommand{\universalBox}[2]{\boxed{\textbf{#1:}\begin{center}\texttt{\large{#2}}\end{center}}}
% some sort of todo box
\newcommand{\todo}[1]{\universalBox{ToDo}{#1}}
% some sort of warning box
\newcommand{\warn}[1]{\universalBox{Warning}{#1}}



%%%%%%%%%%%%%%%%%%%%%%%%%%%%%% document specific %%%%%%%%%%%%%%%%%%%%%%%%%%%%%%

% define the author, title etc
\globalTitle{Interpreter Manual}
\globalSubject{setlX 0.2.2
}
\globalAuthor{Herrmann, Tom}
\globalDate{\today}
\globalKeywords{readme}

% frequently used configuration options
\setboolean{linkHighlighting}{true}

%%%%%%%%%%%%%%%%%%%%%%%%%%%%%%%%%%%% setup %%%%%%%%%%%%%%%%%%%%%%%%%%%%%%%%%%%%

% document layout options
\KOMAoptions{
	listof=totoc, % add list of figures, list of listings etc to toc
	bibliography=totoc, % add bibliography to toc
	abstract=true, % use special formating for abstracts
	parskip=half % use space between paragraphs, don't indent the first line.
}

% enable or disable all indication of links
\ifthenelse{\boolean{linkHighlighting}} {
  % then
	\hypersetup{
		colorlinks=true, % set all link colors to blue
		linkcolor=blue,
		citecolor=red,
		pdfborder={0 0 0} % make sure border around links stays disabled
	}
}{
  % else
	\hypersetup{
		colorlinks=false, % disable link colors
		pdfborder={0 0 0} % make sure border around links stays disabled
	}
}

% set this, if you use nonSections or nonSubsections on the same level as normal numbered sections or subsections
\setboolean{tocAllingnNonChaptersWithNumberedChapters}{false}
\setboolean{tocAllingnNonSectionsWithNumberedSections}{false}

% default settings for listings 
\lstset{
	frame=lines, % frame around the code
	basicstyle=\small\ttfamily, % font size
	numbers=left, % enable line numbering
	numberstyle=\tiny, % font size for numbering
	tabsize=4, % size of tabs in spaces
	breaklines=false, % no line wrap
	xleftmargin=2em, % margin between left page border and listing
	numbersep=1.5em, % margin between numbering and code
	captionpos=b, % caption below the listing
	extendedchars=false % disable special char handling
	showstringspaces=false, % do not show special spaces character in strings
}

% set this to true, if tables should be treated as figures (also added to \listoffigures)
\setboolean{tablesAsFigures}{true}
% set this to true, if listings should be treated as figures (also added to \listoffigures)
\setboolean{listingsAsFigures}{true}

%%%%%%%%%%%%%%%%%%%%% special stuff for setlX project %%%%%%%%%%%%%%%%%%%%%

%%% format aliases for reformation in a context sensitive way

% command example
\newcommand{\command}[1]{\texttt{#1}}

% parameter definition
\newcommand{\param}[2]{
	\item[] \command{#1}\\
            \begin{tabular}{ l p{30 em} }
                   \hspace*{1 em} & #2 \\
            \end{tabular}
}

% folder description
\newcommand{\folder}[2]{
	\item \command{#1}\\
            \begin{tabular}{ l p{30 em} }
                   \hspace*{1 em} & #2 \\
            \end{tabular}
}

% setlX and SetlX
\def\setlX{\textsc{setlX}}    % setlX --- the interpreter
\def\SetlX{\textsc{SetlX}}    % SetlX --- the language
\def\Setl{\textsc{Setl}}      % Setl  --- the language
\def\SetlTwo{\textsc{Setl2}}  % Setl2 --- the language


%%%%%%%%%%%%%%%%%%%%%%%%%%%%%%%%%%%%%%%%%%%%%%%%%%%%%%%%%%%%%%%%%%%%%%%%%%%%%%%
%%%%%%%%%%%%%%%%%%%%%%%%%%%%%%%% document text %%%%%%%%%%%%%%%%%%%%%%%%%%%%%%%%
%%%%%%%%%%%%%%%%%%%%%%%%%%%%%%%%%%%%%%%%%%%%%%%%%%%%%%%%%%%%%%%%%%%%%%%%%%%%%%%

\begin{document}
\begin{titlepage}
\maketitle

\vfill

\begin{center}
\Large
``Computers make very fast, very accurate mistakes.''
\end{center}

\vfill
\tableofcontents
\end{titlepage}

\section{Overview}

This is the manual of \setlX, an interpreter for the \SetlX{} (\underline{set} \underline{l}anguage e\underline{x}tended) programming-language.

The \SetlX{} language was mostly designed by Prof. Dr. Karl Stroetmann and is an evolution of \Setl{} by Jack Schwartz. It was specifically conceived to make the unique features of \Setl{} more accessible to today's students of set theories.

The \setlX{} interpreter, which currently is the \SetlX{} reference implementation, was implemented by Tom Herrmann. Its official homepage is \url{http://setlX.randoom.org/}. You may send bug-reports and\slash{}or questions about the \setlX{} interpreter via e-mail to \mailto{setlx@randoom.org}.


\section{Installation}

To comfortably use \setlX{} it may be installed into the search-path. After this is done, it may be launched just like any other command-line program on your system.

\subsection{Unix-like OS (Linux, MacOS X, etc.)}\label{Unix}

\begin{enumerate}
	\item Copy the `\command{setlX.jar}'-file into some folder, which is accessible by all users, who should be able to execute \setlX{} (e.g. `\command{/usr/local/setlX/}').\\
	You may also copy it into the home directory of some user, when he is the only \setlX-user on the system (e.g. `\command{/home/<user>/programs/setlX/}').
	\item Open the `\command{setlX}'-script in your favorite editor (either `\command{vi}' or `\command{emacs}' of course).
	\item Change the contents of the `\command{setlXJarLocation}'-variable to the full path of the copied jar-file, including the filename itself, e.g.
\begin{lstlisting}[frame=none,numbers=none]
setlXJarLocation="/usr/local/setlX/setlX.jar"
\end{lstlisting}
	\item Save and close the file.
	\item Copy the modified `\command{setlX}'-script into some folder, which is in the search-path of all users which should be able to execute \setlX{} (e.g. `\command{/usr/local/bin}').\\
	When only used by one user, you may copy it into the `bin' directory in his home instead (e.g. `\command{/home/<user>/bin}').
	\item Make the copied script executable, e.g.
\begin{lstlisting}[frame=none,numbers=none]
chmod +x /usr/local/bin/setlX
\end{lstlisting}
	\item \textbf{(Optional)} Set the `\command{SETLX\_LIBRARY\_PATH}' environment variable to the path where \setlX{} should look for library files, e.g. add the line
\begin{lstlisting}[frame=none,numbers=none]
export SETLX_LIBRARY_PATH="/home/<user>/setlX/library";
\end{lstlisting}
to the configuration file of your shell (e.g. `\command{/home/<user>/.bashrc}').
\end{enumerate}

\subsection{Microsoft Windows}

\begin{center}
\large
``Here's a nickel kid. Go buy yourself a real computer.''\footnote{In reference to \url{http://dilbert.com/1995-06-24}}
\end{center}

\begin{enumerate}
	\item Copy the `\command{setlX.jar}' and `\command{setlX.cmd}' files into some folder, which is accessible by all users, who should be able to execute \setlX{} (e.g. `\command{C:\textbackslash{}Progam Files\textbackslash{}setlX\textbackslash{}}').
	\item Open the copied `\command{setlX.cmd}'-script in your favorite text editor.
	\item Change the contents of the `\command{setlXJarLocation}'-variable to the full path of the jar-file, including the filename itself, e.g.
\begin{lstlisting}[frame=none,numbers=none]
set setlXJarLocation=C:\Progam Files\setlX\setlX.jar
\end{lstlisting}
	\item Save and close the file.
	\item Add the folder you placed both files into to the search-path (requires Administrator privileges):
	\begin{enumerate}
		\item Press \command{[Win]+[Pause]} key combination to open system preferences.
		\item 
		\begin{itemize}
			\item On Windows Vista or newer:\\Click on `\command{Advanced system settings}' on the left pane, which opens a new window.
			\item On Windows XP or older:\\Click the `\command{Advanced}' tab in the current window.
		\end{itemize}
		\item Click on `\command{Environment Variables}' on the bottom, which opens another window.
		\item Select the `\command{Path}' variable in the `\command{System variables}' section and click on `\command{Edit...}', which opens jet another window.
		\item Do not alter the existing content of the `\command{Variable value}' field, but add a semicolon (`\command{;}'), followed by the full path to the folder where you installed setlX, to its very end. Using the previous example path, the new value should look like:
\begin{lstlisting}[frame=none,numbers=none]
<previous value>;C:\Progam Files\setlX
\end{lstlisting}
		\item \textbf{(Optional)} Add the `\command{SETLX\_LIBRARY\_PATH}' environment variable and set it to the path where \setlX{} should look for library files, e.g.
\begin{lstlisting}[frame=none,numbers=none]
C:\Progam Files\setlX\library
\end{lstlisting}
		\item Accept all changes by clicking `\command{OK}' in all windows you opened before.
		\item End you current session and login into Windows again.
	\end{enumerate}
\end{enumerate}

\newpage

\section{Usage}

The \setlX{} interpreter has two basic modes of operation:

\begin{itemize}
	\item file execution mode, which can be launched with
\begin{lstlisting}[frame=none,numbers=none]
setlX <path>/<name>.stlx
\end{lstlisting}
	\item interactive mode, which can be launched with
\begin{lstlisting}[frame=none,numbers=none]
setlX
\end{lstlisting}
\end{itemize}

The `file execution mode' will be started if one or more paths to code files are supplied as parameters for this program. These files will then be parsed and executed in the order in which they are listed as parameters. When executing multiple files, they share the outermost scope.

When no paths are supplied, the `interactive mode' will be started.\\
Executing the `\command{exit;}' statement will terminate the interpreter.

\nonSubsection{Additional parameters}

In addition to file-paths a number of options can be used when running \setlX{}:

\begin{itemize}
	\param{--help}
	      {displays some helpful information}
	\param{--libraryPath <path>}
	      {override SETLX\_LIBRARY\_PATH environment variable}
	\param{--multiLineMode}
	      {only accept input in interactive mode after additional new line}
	\param{--noAssert}
	      {disables all assert functions}
	\param{--noExecution}
	      {load and check code for syntax errors, but do not execute it}
	\param{--params <argument> ...}
	      {passes all following arguments to the executed program via `params' variable}
	\param{--predictableRandom}
	      {always returns the same pseudo random sequence of numbers from the internal random number generator (for debugging)}
	\param{--real32\\
	       --real64\\
	       --real128\\
	       --real256}
           {sets the width of the real-type in bits (64 is the sane default)}
	\param{--verbose}
           {display the parsed program before executing it (has no effect in interactive mode)}
	\param{--version}
           {displays the interpreter version and terminates}
\end{itemize}

\section{Building from source}

Optionally, when the interpreter source is available, the interpreter can be (re)build by executing

\begin{lstlisting}[frame=none,numbers=none]
make
\end{lstlisting}

in the `\command{src}' folder.

A self-contained `.jar' file will be created, which can be launched on all Java-Platforms without additional scripts, jars or environment-variables by executing

\begin{lstlisting}[frame=none,numbers=none]
java -jar setlX.jar <path>/<name>.stlx
\end{lstlisting}

or

\begin{lstlisting}[frame=none,numbers=none]
java -jar setlX.jar
\end{lstlisting}

However, the \command{setlX} and \command{setlX.cmd} scripts can also be used to launch the jar file created by this process.

\subsection{Building for Java 1.5 (esp. a 1st Generation Intel Macintosh)}\label{MACj5}

To create \setlX{} for a machine, which is limited to Java 1.5, you need another machine running both Java 1.5 and Java 1.6.

The main example for such situation is a 1st generation Apple Macintosh with Intel CPU running Mac OS X 10.5 or earlier. These machines are artificially limited by Apple to Java 1.5 only. It is recommended that you try installing an unofficial ``Soylatte''\footnote{\url{http://landonf.bikemonkey.org/static/soylatte/}} port of Java 1.6 for these machines, or upgrade to Mac OS X 10.6 or higher. You should only attempt to rebuild \setlX{} for Java 1.5 if you can not, or do not want to install this port.

On the second machine running Java 1.5 and 1.6, do the following:

\begin{enumerate}
	\item Activate Java 1.6.\\
		On a Mac this can be done using:
\begin{lstlisting}[frame=none,numbers=none]
/Applications/Utilities/Java Preferences
\end{lstlisting}
	\item Switch to the sub-directory src and execute the following command:
\begin{lstlisting}[frame=none,numbers=none]
make java-src/grammar/SetlXgrammar*.java
\end{lstlisting}
	\item Activate Java 1.5.\\
		On a Mac this can again be done using:
\begin{lstlisting}[frame=none,numbers=none]
/Applications/Utilities/Java Preferences
\end{lstlisting}
	\item Switch to the sub-directory src again and execute the following command:
\begin{lstlisting}[frame=none,numbers=none]
make
\end{lstlisting}
\end{enumerate}
This will create the file `\command{setlX.jar}', which can be installed on the machine running Java 1.5 as stated in section \ref{Unix}.

\section{System Requirements}

To run \setlX{} a Java runtime (JRE) is required, which is compatible to Java version 1.6 (aka version 6) or higher. Running it with Version 1.5 (aka version 5) is largely untested, but should work. However recompilation from source is required and somewhat problematic (see section \ref{MACj5}).

When building from source, the corresponding Java Development Kit (JDK) has to be present as well.

The hardware requirements are highly dependent on the executed \SetlX{} program. For optimal performance in most situations at least 512 MB of free main-memory should be available.

\section{Limitations}

\subsection{Stack Overflow \slash{} Out of Memory Error}

Due to its implementation and execution by the Java Virtual Machine (JVM), a `stack overflow' error may be encountered when executing highly recursive \SetlX{} programs. An `out of memory error' might also be encountered when using very large or very many sets and\slash{}or lists.

The JVM is unable to dynamically increase its internal stack size at all and the heap space only in very limited bounds, no matter how much free memory is available.

To work around these problems, edit the used launching script (`\command{setlX}' or `\command{setlX.cmd}') with a text editor and follow the directions given at the top.

\subsection{Subpar Parsing Errors}

Parsing errors are subpar, which make root cause of syntax errors sometimes hard to detect in a slew of messages.

\subsection{Input in Interactive Mode on Unix}

On UNIX systems, the prompt in interactive mode does not handle control sequences correctly. This is a Java Limitation and all no platform independent solutions are available. You may use the `rlwrap' program to work around this issue.

%\section{Known Bugs}
%The following bugs are known to be present in this version of the interpreter:
%
%\begin{itemize}
%	\item ...
%\end{itemize}


\section{Disclaimer}
You may distribute \setlX{}, to anyone and for any purpose, and make changes to it, even without permission from the author or notice to the author, as long as you do not remove any copyright information, and also comply with the ANTLR license.

This program uses the ANTLR parser generator in version 3.4. As per its license ANTLR is not guaranteed to work and might even destroy all life on this planet. The full text of this license is shown in figure \ref{lblAntlr}. Note however, that it only applies to the ANTLR components and NOT to the rest of the interpreter.

\includeListing[e]{../src/antlr/antlr_LICENSE.txt}{}{lblAntlr}{}{License used by ANTLR v3.4}


\end{document}
