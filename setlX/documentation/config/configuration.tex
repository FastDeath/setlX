% document layout options
\KOMAoptions{
	listof=totoc, % add list of figures, list of listings etc to toc
	bibliography=totoc, % add bibliography to toc
	abstract=true, % use special formating for abstracts
	parskip=half % use space between paragraphs, don't indent the first line.
}

% enable or disable all indication of links
\ifthenelse{\boolean{linkHighlighting}} {
  % then
	\hypersetup{
		colorlinks=true, % set all link colors to blue
		linkcolor=blue,
		citecolor=red,
		pdfborder={0 0 0} % make sure border around links stays disabled
	}
}{
  % else
	\hypersetup{
		colorlinks=false, % disable link colors
		pdfborder={0 0 0} % make sure border around links stays disabled
	}
}

% set this, if you use nonSections or nonSubsections on the same level as normal numbered sections or subsections
\setboolean{tocAllingnNonChaptersWithNumberedChapters}{false}
\setboolean{tocAllingnNonSectionsWithNumberedSections}{false}

% default settings for listings 
\lstset{
	frame=lines, % frame around the code
	basicstyle=\small\ttfamily, % font size
	numbers=left, % enable line numbering
	numberstyle=\tiny, % font size for numbering
	tabsize=4, % size of tabs in spaces
	breaklines=false, % no line wrap
	xleftmargin=2em, % margin between left page border and listing
	numbersep=1.5em, % margin between numbering and code
	captionpos=b, % caption below the listing
	extendedchars=false % disable special char handling
}

% set this to true, if tables should be treated as figures (also added to \listoffigures)
\setboolean{tablesAsFigures}{true}
% set this to true, if listings should be treated as figures (also added to \listoffigures)
\setboolean{listingsAsFigures}{true}

%%%%%%%%%%%%%%%%%%%%% special stuff for setlX project %%%%%%%%%%%%%%%%%%%%%

%%% format aliases for reformation in a context sensitive way

% command example
\newcommand{\command}[1]{\texttt{#1}}

% parameter definition
\newcommand{\param}[2]{
	\item[] \command{#1}\\
            \begin{tabular}{ l p{30 em} }
                   \hspace*{1 em} & #2 \\
            \end{tabular}
}

% folder description
\newcommand{\folder}[2]{
	\item \command{#1}\\
            \begin{tabular}{ l p{30 em} }
                   \hspace*{1 em} & #2 \\
            \end{tabular}
}

% setlX and SetlX
\def\setlX{\textsc{setlX}}    % setlX --- the interpreter
\def\SetlX{\textsc{SetlX}}    % SetlX --- the language
\def\Setl{\textsc{Setl}}      % Setl  --- the language
\def\SetlTwo{\textsc{Setl2}}  % Setl2 --- the language
