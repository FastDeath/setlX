\chapter{Introduction}
Every year, dozens of new programming languages are proposed and each time this
happens,  the inventors of the new language have to answer the same question: 
There are hundreds of programming lanuages already, why do we need yet another one?  
Of course, the answer is always the same.  It consists of an economical argument, a
theological argument and a practical argument. For the
convenience of the reader, let us brievely review these arguments as they read for \setlx.  
\begin{enumerate}
\item Nothing less than the prosperity and wellfare of the entire universe is at
      stake and only \setlx\ provides the means to save it.
\item Programming in \setlx\ is the only way to guarantee redemption from 
      the eternal hell fire that awaits those not programming in \setlx.
\item Programming in \setlx\ is fun!
\end{enumerate}
The economical argument has already been discussed at length by Adams \cite{adams:1980},
therefore we don't have to repeat it here.   We deeply
regret the fact that the philosophical background of the average computer scientist does not permit
them to follow advanced theological discussions.  Therefore, we refrain from giving a
detailed proof of the second claim.  Nevertheless, we hope the examples given in this
tutorial will convince the reader of the truth of the third claim.
  One of the reasons for this is that \setlx\ programs are both very concise and
readable.  This often makes it possible to fit the implementation of complex algorithms in \setlx\ on a
single slide because the \setlx\ program is as concise as the pseudocode that is usually used to
present algorithms in lectures.  The benefit of this is that instead of pseudocode, students have a
running program.  Indeed, the conciseness of \setlx\ programs was one of the reasons for
the first author to adopt \setlx\ as a programming language in his courses on computer
science: It is often feasible to write a complete \setlx\ program in a few lines 
onto the blackboard, since \setlx\ programs are as concise as mathematical formulae.


\setlx\ is well suited to implement complex algorithms. This is achieved because SetlX
provides a number of sophisticated builtin data types that enable the user to code at a very high
abstraction level.
These types are sets, lists, first-order terms, and functions.  
As sets are implemented as ordered binary trees, sets of pairs can be used both as symbol
tables and as priority queues.  This enables a very concise implementation 
of a number of graph theoretical algorithms.

The purpose of this tutorial is to introduce the most important features of \setlx\ and to
show how the use of the above mentioned data types leads to programs that are both shorter
and clearer than the corresponding programs in other programming languages.  This was the
prime motivation of the first author to develop \setlx:  It turns out that \setlx\ is very
convenient as a tool to present algorithms at a high abstraction level in a class room. 
Furthermore, \setlx\ makes the abstract concepts of set theory tangible for students of
computer science. 
\pagebreak

\noindent 
The remainder
of this tutorial is structured as follows:
\begin{enumerate}
\item In the second chapter, we discuss the data types available in \setlx.
\item The third chapter provides the control structures.
\item The fourth chapter deals with regular expressions.
\item The fifth chapter discusses the \texttt{try-}\texttt{catch} and \texttt{throw}
      mechanism and demonstrates the use of \emph{backtracking}.
\item The final chapter explains the predefined functions.
\end{enumerate}
This tutorial is not meant as an introduction to programming.  It assumes that the reader
has had some preliminary exposure to programming and has already written a few programs in
either \texttt{C}, \textsl{Java}, or a similar language.

\section*{Downloading}
The current distribution of  \setlx\ can be downloaded from either
\\[0.2cm]
\hspace*{1.3cm}
\href{http://wwwlehre.dhbw-stuttgart.de/~stroetma/SetlX/setlX.php}{\texttt{http://wwwlehre.dhbw-stuttgart.de/\symbol{126}stroetma/SetlX/setlX.php}}
\\[0.2cm]
or
\\[0.2cm]
\hspace*{1.3cm}
\href{http://randoom.org/Software/SetlX}{\texttt{http://randoom.org/Software/SetlX}}.
\\[0.2cm]
\setlx\ is
written in \textsl{Java} and is therefore supported on a number of different operating
systems.  Currently, \setlx\ is supported on \textsl{Linux}, \textsl{Mac OS X},
\textsl{Microsoft Windows}, and \textsl{Android}.

The websites given above explain how to install the language on various platforms.  
The distribution contains the \textsl{Java} code and a development guide that gives an
overview of the implementation. 

\subsection*{Disclaimer}
The development of \setlx\ is an ongoing project.  Therefore some of the material presented in
this tutorial might be out of date, while certain aspects of the language won't be
covered.  The current version of this tutorial is not intended to be a reference manual.
The idea is rather to provide the reader with an introduction that is sufficient to get started.

\subsection*{Encouragement}
The autors would be grateful for any kind of feedback.  The authors can be contacted via
email as follows:
\begin{tabbing}
\qquad \= Karl Stroetmann: \qquad \= \href{mailto:stroetmann@dhbw-stuttgart.de}{\texttt{stroetmann@dhbw-stuttgart.de}} \\[0.2cm]
       \> Tom Herrmann:           \> \href{mailto:setlx@randoom.org}{\texttt{setlx@randoom.org}}
\end{tabbing}

\subsection*{Acknowledgements}
The autors would like to acknowledge that Karl-Friedrich Gebhardt and Hakan Kjellerstrand 
have both  read earlier drafts of
this tutorial and have given valuable feedback that has helped to improve the current
presentation. 

%%% Local Variables: 
%%% mode: latex
%%% TeX-master: "tutorial"
%%% End: 
