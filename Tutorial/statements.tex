\chapter{Statements}
This section discusses the various possibilities to write statements and the 
features offered by \setlx\ to steer the control flow in a program.
\setlx\ supports the following means to control the order of statement execution:
\begin{enumerate}
\item branching statements like \texttt{if-then-else}, \texttt{switch}, and \texttt{match},
\item the looping statements \texttt{for} and \texttt{while} together with \texttt{break} and
      \texttt{continue}, 
\item \texttt{try-catch} statements to deal with exceptions,
\item the \texttt{backtrack} statement to support backtracking.
\end{enumerate}

\section{Assignment Statements}
The most basic command is the assignment.  In contrast to the programming languages \texttt{C}
and \textsl{Java}, \setlx\ uses the operator ``\texttt{:=}'' for assignments.  For example, the statement
\\[0.2cm]
\hspace*{1.3cm}
\texttt{x := 2/3;}
\\[0.2cm]
binds the variable \texttt{x} to the fraction $\frac{2}{3}$.  \setlx\ supports simultaneous
assignments to multiple variables via lists.  For example, the statement
\\[0.2cm]
\hspace*{1.3cm}
\texttt{[x, y, z] := [1, 2, 3];}
\\[0.2cm]
simultaneously binds the variables \texttt{x} to $1$, \texttt{y} to $2$ and \texttt{z} to $3$.
This feature can be used to swap the values of two variables:  The statement
\\[0.2cm]
\hspace*{1.3cm}
\texttt{[x, y] := [y, x];}
\\[0.2cm]
swaps the values of $x$ and $y$.  If we do not need to assign all the values of a list, we
can use the underscore ``\texttt{\_}'' as an anonymous variable.  For example,
\\[0.2cm]
\hspace*{1.3cm}
\texttt{[x, \_, z] := [1, 2, 3];} 
\\[0.2cm]
assigns the number \texttt{1} to the variable \texttt{x} and the variable \texttt{z} is
is set to \texttt{3}.

The assignment operator can be combined with any of the operators 
``\texttt{+}'',
``\texttt{-}'',
``\texttt{*}'',
``\texttt{/}'',
``\texttt{\symbol{37}}'', and 
``\texttt{\symbol{92}}''.
For example, the statement
\\[0.2cm]
\hspace*{1.3cm}
\texttt{x += 1;}
\\[0.2cm]
increments the value of the variable \texttt{x} by one,  while the statement
\\[0.2cm]
\hspace*{1.3cm}
\texttt{x *= 2;}
\\[0.2cm]
doubles the value of \texttt{x}.   Finally,
assignment statements can be chained.  For example, the statement
\\[0.2cm]
\hspace*{1.3cm}
\texttt{a := b := 3;}
\\[0.2cm]
assigns the value \texttt{3} to both  \texttt{a} and \texttt{b}.

\section{Functions}
The code shown in figure
\ref{fig:primes-slim.stlx} on page \pageref{fig:primes-slim.stlx} shows a simple program
to compute prime numbers.  It defines two functions.  The function \texttt{factors} takes a
natural number $p$ as its first argument and computes the set of all factors of $p$.  Here, a number
$f$ is a factor of $p$ iff dividing $p$ by $f$ leaves no remainder, that is 
$p \,\texttt{\symbol{37}}\, f = 0$.
The second function \texttt{primes} takes a natural number $n$ as an argument and computes the
set of all those prime numbers that are less or equal to $n$.  

The idea is that a number $p$ is prime
iff the set of all  factors of $p$ contains just $1$ and $p$.  
Note that, as \setlx\ is a functional language, the functions that are defined by the
keyword \texttt{procedure}  are assigned to variables.  As already
mentioned n the previous chapter, these functions can be used like
any other values.


\begin{figure}[!ht]
\centering
\begin{Verbatim}[ frame         = lines, 
                  framesep      = 0.3cm, 
                  firstnumber   = 1,
                  labelposition = bottomline,
                  numbers       = left,
                  numbersep     = -0.2cm,
                  xleftmargin   = 0.8cm,
                  xrightmargin  = 0.8cm,
                ]
    factors := procedure(p) {
        return { f : f in { 1 .. p } | p % f == 0 };
    };
    primes := procedure(n) {
        return { p : p in { 2 .. n } | factors(p) == { 1, p } };
    };
    print(primes(100));
\end{Verbatim}
\vspace*{-0.3cm}
\caption{A naive program to compute primes.}
\label{fig:primes-slim.stlx}
\end{figure}

\noindent
A simplified grammar rule for the definition of a function can be given as follows:
\\[0.2cm]
\hspace*{1.3cm}
$\textsl{fctDef} \;\rightarrow\; \mathtt{VAR}\; \quoted{:=} \quoted{procedure} \quoted{(} \textsl{paramList} 
 \quoted{)} \quoted{\{} \textsl{block} \quoted{\}} \quoted{;}$
\\[0.2cm]
The meaning of the symbols used in this grammar rule are as follows:
\begin{enumerate}
\item \texttt{VAR} identifies a variable.  This variable is bound to the definition of the function.
      Note that in \setlx\ the name of a variable has to start with a lower case letter.
\item \textsl{paramList} is a list of the formal parameters of the function.  In
      \textsc{Ebnf}-notation the grammar rule for \textsl{paramList} is given as
      \\[0.2cm]
      \hspace*{1.3cm}
      $\textsl{paramList} \;\rightarrow\; (\textsl{paramSpec}\;\; (\squoted{,}\; \textsl{paramSpec})^*)?$
      \\[0.2cm]
      Therefore, a \textsl{paramList} is a possibly empty list of parameter specifications that are 
      separated by a comma ``\texttt{,}''.  A parameter specification is either just a variable
      or it is a variable preceded by the token ``\texttt{rw}'':
      \\[0.2cm]
      \hspace*{1.3cm}
      $\textsl{paramSpec} \;\rightarrow\; (\mathtt{ID} \mid \squoted{rw}\; \mathtt{ID})$.
      \\[0.2cm]  
      If the parameter is preceded by the keyword \texttt{rw}, then this parameter is a 
      \emph{read-write parameter}, which means that the function can change the value of the
      variable given as argument and this change will then be visible
      outside of the function.  Therefore, parameters prefixed with
      the keyword \texttt{rw} have a 
      \href{http://en.wikipedia.org/wiki/Call_by_name#Call_by_name}{\emph{call by name}}
      semantics.  Parameters not specified as read-write parameters have a strict
      \href{http://en.wikipedia.org/wiki/Call_by_value#Call_by_value}{\emph{call by value}} 
      semantics, and therefore changes to those parameters will not be
      visible outside the function.
\item \textsl{block} is a sequence of statements.
\end{enumerate}
Note that the definition of a function has to be terminated by the symbol ``\texttt{;}''.

There is a variant syntax for defining a function which is appropriate if the definition of the
function is just a single expression.  For example, the function mapping $x$ to the square
$x*x$ can be defined as
\\[0.2cm]
\hspace*{1.3cm}
\texttt{f := x |-> x * x;}
\\[0.2cm]
A definition of this form is called a \emph{lambda definition}.
If the function takes more than one arguments, the argument list has to be enclosed in square
brackets.  For example, the function \texttt{hyp} that computes the hypotenuse of a rectangular triangle
can be defined as follows:
\\[0.2cm]
\hspace*{1.3cm}
\texttt{hyp := [x, y] |-> sqrt(x*x + y*y);}
\\[0.2cm]
The syntax for a lambda definition is given by the follwing grammar rule:
\\[0.2cm]
\hspace*{1.3cm}
$\textsl{fctDef} \;\rightarrow\; \mathtt{ID} \quoted{:=} \textsl{lambdaParams} \quoted{|->} \textsl{expr} \quoted{;}$
\\[0.2cm]
Here, \textsl{lambdaParams} is either just a single parameter or a list of parameters, where the
parameters are enclosed in square brackets and are separated by commas, while
\textsl{expr} denotes an expression.

Lambda definitions are handy if we don't bother to give a name to a function.
For example, the code in figure \ref{fig:lambda.stlx} defines a function map.  This function takes
two arguments: The first argument $l$ is a list and the second argument $f$ is a function that is to
be applied to all arguments of this list.  In line 4, the function map is called with a function
that squares its argument.  Therefore, the assignment in line 4 computes the list of the first 10
square numbers.  Note that we did not had to name the function that did the squaring.  Instead, we
have  used a lambda definition.

\begin{figure}[!ht]
\centering
\begin{Verbatim}[ frame         = lines, 
                  framesep      = 0.3cm, 
                  firstnumber   = 1,
                  labelposition = bottomline,
                  numbers       = left,
                  numbersep     = -0.2cm,
                  xleftmargin   = 0.8cm,
                  xrightmargin  = 0.8cm,
                ]
    map := procedure(l, f) {
        return [ f(x) : x in l ];
    };    
    t := map([1 .. 10], x |-> x * x);
\end{Verbatim}
\vspace*{-0.3cm}
\caption{An example of a lambda definition in use.}
\label{fig:lambda.stlx}
\end{figure}

\noindent
Of course, it is much easier to build the list of the first 10 squares using the statement
\\[0.2cm]
\hspace*{1.3cm}
\texttt{t := [x*x : $\hspace*{-0.3cm}$x in [1..10]];}

\subsection{Memoization}
The function $\textsl{fib}: \mathbb{N} \rightarrow
\mathbb{N}$ computing the
\href{http://en.wikipedia.org/wiki/Fibonacci_numbers}{\emph{Fibonacci numbers}}
is defined recursively by the following set of recurrence equations:
\\[0.2cm]
\hspace*{1.3cm}
$\textsl{fib}(0) = 0$, \quad
$\textsl{fib}(1) = 1$, \quad and \quad
$\textsl{fib}(n+2) = \textsl{fib}(n+1) + \textsl{fib}(n)$. 
\\[0.2cm]
These equations are readily implemented as shown in Figure \ref{fig:fibonacci.stlx}.
However, this implementation has a performance problem which can be easily seen when tracing the
computation of $\textsl{fib}(4)$.  


\begin{figure}[!ht]
\centering
\begin{Verbatim}[ frame         = lines, 
                  framesep      = 0.3cm, 
                  firstnumber   = 1,
                  labelposition = bottomline,
                  numbers       = left,
                  numbersep     = -0.2cm,
                  xleftmargin   = 0.8cm,
                  xrightmargin  = 0.8cm,
                ]
    fibonacci := procedure(n) {
        if (n in [0,1]) {
            return n;
        }
        return fibonacci(n-1) + fibonacci(n-2);
    };
\end{Verbatim}
\vspace*{-0.3cm}
\caption{A naive implementation of the Fibonacci function.}
\label{fig:fibonacci.stlx}
\end{figure}

\begin{figure}[!ht]
\centering
\begin{Verbatim}[ frame         = lines, 
                  framesep      = 0.3cm, 
                  firstnumber   = 1,
                  labelposition = bottomline,
                  numbers       = left,
                  numbersep     = -0.2cm,
                  xleftmargin   = 0.8cm,
                  xrightmargin  = 0.8cm,
                ]
    fibonacci := procedure(n) {
        if (n in [0,1]) {
            result := n;
        } else {
            result := fibonacci(n-1) + fibonacci(n-2);
        }
        print("fibonacci($n$) = $result$");
        return result;
    };
\end{Verbatim}
\vspace*{-0.3cm}
\caption{Tracing the computation of the Fibonacci function.}
\label{fig:fibonacci-trace.stlx}
\end{figure}

In order to trace the computation, we change the program as shown in Figure
\ref{fig:fibonacci-trace.stlx}.  
If we evaluate the expression \texttt{fibonacci(4)}, we get the output shown in Figure
\ref{fig:fibonacci.trace}.  This output shows that the expression \texttt{fibonacci(2)} is
evaluated twice.  The reason is that the value of \texttt{fibonacci(2)} is needed in the equation
\\[0.2cm]
\hspace*{1.3cm}
\texttt{fibonacci(4)} := \texttt{fibonacci(3)} + \texttt{fibonacci(2)}
\\[0.2cm]
to compute \texttt{fibonacci(4)}, but then in order to compute \texttt{fibonacci(3)}, we
have to compute \texttt{fibonacci(2)} again.  The trace also shows that the problem gets
aggravated the longer the computation runs.  For example, the value of
\texttt{fibonacci(1)} has to be computed three times.

\begin{figure}[!ht]
\centering
\begin{Verbatim}[ frame         = lines, 
                  framesep      = 0.3cm, 
                  firstnumber   = 1,
                  labelposition = bottomline,
                  numbers       = none,
                  numbersep     = -0.2cm,
                  xleftmargin   = 0.8cm,
                  xrightmargin  = 0.8cm,
                ]
    => fibonacci(4);
    
    fibonacci(1) = 1
    fibonacci(0) = 0
    fibonacci(2) = 1
    fibonacci(1) = 1
    fibonacci(3) = 2
    fibonacci(1) = 1
    fibonacci(0) = 0
    fibonacci(2) = 1
    fibonacci(4) = 3
    ~< Result: 3 >~
    
    => 
\end{Verbatim}
\vspace*{-0.3cm}
\caption{Output of evaluating the expression \texttt{fibonacci(4)}.}
\label{fig:fibonacci.trace}
\end{figure}


In order to have a more efficient computation, it is necessary to memorize the values of
the function \texttt{fibonacci} once they are computed.  Fortunately, \setlx\ offers
\emph{cached functions}.  If a function $f$ is declared as a cached function, then every
time the function $f$ is evaluated for an argument $x$, the computed value $f(x)$ is
memorized and stored in a table.  The next time the function $f$ is used to compute
$f(x)$, the interpreter first checks whether the value of $f(x)$ has already been
computed.  In this case, instead of computing $f(x)$ again,  the function returns the
value stored in the table.  This technique is known as
\href{http://en.wikipedia.org/wiki/Memoization}{\emph{memoization}}.
Fortunately, memoization is directly supported in \setlx\ via cached functions.
Figure
\ref{fig:fibonacci-cached.stlx} shows an implementation of the Fibonacci function as a
cached function.  If we compare the program in Figure
\ref{fig:fibonacci-cached.stlx} with our first attempt shown in Figure
\ref{fig:fibonacci.stlx}, then we see that the only difference is that instead of the
keyword ``\texttt{procedure}'' we have used the keyword ``\texttt{cachedProcedure}'' instead.


\begin{figure}[!ht]
\centering
\begin{Verbatim}[ frame         = lines, 
                  framesep      = 0.3cm, 
                  firstnumber   = 1,
                  labelposition = bottomline,
                  numbers       = left,
                  numbersep     = -0.2cm,
                  xleftmargin   = 0.8cm,
                  xrightmargin  = 0.8cm,
                ]
    fibonacci := cachedProcedure(n) {
        if (n in [0,1]) {
            return n;
        } 
        return fibonacci(n-1) + fibonacci(n-2);
    };
\end{Verbatim}
\vspace*{-0.3cm}
\caption{A cached implementation of the Fibonacci function.}
\label{fig:fibonacci-cached.stlx}
\end{figure}

\vspace*{0.3cm}

\noindent
\textbf{Warning}:  A function should only be declared as a \texttt{cachedProcedure} if it is
guaranteed to always produce the same result when called with the same argument.
Therefore, a function should not be declared as a \texttt{cachedProcedure} if it does one
of the following things:
\begin{enumerate}
\item The function makes use of random numbers.
\item The function reads input either from a file or from the command line.
\end{enumerate}
To further support cached procedures,  \setlx\ provides the function \texttt{cacheStats},
which is called with a single argument that must be a cached function.  For example, if
we define the function \texttt{fibonacci} as shown in Figure
\ref{fig:fibonacci-cached.stlx} and evaluate the expression \texttt{fibonacci(100)}, then
the expression \texttt{cacheStats(fib)} gives the following result:
\\[0.2cm]
\hspace*{1.3cm}
\texttt{\symbol{126}< Result: {["cache hits", 98], ["cached items", 101]} >\symbol{126}}
\\[0.2cm]
This tells us that the cache contains 101 different argument/value pairs, as the cache
now stores the values
\\[0.2cm]
\hspace*{1.3cm}
\texttt{fibonacci($n$)} \quad for all $n \in \{0,\cdots,100\}$.
\\[0.2cm]
Furthermore, we see that 98 of these 101 argument/value pairs have
been used more than once in order to
compute the values of \texttt{fibonacci} for different arguments.  The reason is that the
values for the last three arguments 99, 100, and 101 have not yet been used for the
computation of different values, but all other arguments have been used at least once for
computing another value.

In order to prevent memory leaks, \setlx\ provides the function \texttt{clearCache}.  This
function is invoked with one argument which must be a cached function.  Writing
\\[0.2cm]
\hspace*{1.3cm}
\texttt{clearCache(f)}
\\[0.2cm]
clears the cache for the function $f$, that is all argument/value pairs stored for $f$
will be removed from the cache.


\section{Branching Statements}
Like most modern languages, \setlx\ supports \texttt{if-then-else} statements and
\texttt{switch} statements.  A generalization of \texttt{switch} statements, the so called
\texttt{match} statements, are also supported. We begin our discussion with
\texttt{if-then-else} statements.

\subsection{\texttt{if-then-else} Statements}
In order to support branching, \setlx\ supports \texttt{if-then-else} statements.  The syntax is 
similar to the corresponding syntax in the programming language \texttt{C}.  However, braces are
required.  For example, figure \ref{fig:toBin.stlx} on page \pageref{fig:toBin.stlx} shows a
recursive function that computes the binary representation of a
natural number.  Here, the function \texttt{str} converts its argument
into a string while the function $\texttt{floor}(x)$ computes the largest
natural number that is less or equal to its argument $x$.

\begin{figure}[!ht]
\centering
\begin{Verbatim}[ frame         = lines, 
                  framesep      = 0.3cm, 
                  firstnumber   = 1,
                  labelposition = bottomline,
                  numbers       = left,
                  numbersep     = -0.2cm,
                  xleftmargin   = 0.8cm,
                  xrightmargin  = 0.8cm,
                ]
    toBin := procedure(n) {
        if (n < 2) {
            return str(n);
        } else {
            r := n % 2;
            n := floor(n / 2);
            return toBin(n) + toBin(r);
        }
    };
\end{Verbatim}
\vspace*{-0.3cm}
\caption{A function to compute the binary representation of a natural number.}
\label{fig:toBin.stlx}
\end{figure}

As in the programming languages \texttt{C} and \textsl{Java}, the
\texttt{else} clause is optional.

\subsection{\texttt{switch} Statements}
Figure \ref{fig:sort3.stlx} shows a function that takes a list of length 3.  The function sorts
the resulting list.  In effect, the function \texttt{sort3} implements a 
\href{http://en.wikipedia.org/wiki/Decision_tree}{\emph{decision tree}}.
This example shows how \texttt{if-then-else} statements can be cascaded.

\begin{figure}[!ht]
\centering
\begin{Verbatim}[ frame         = lines, 
                  framesep      = 0.3cm, 
                  firstnumber   = 1,
                  labelposition = bottomline,
                  numbers       = left,
                  numbersep     = -0.2cm,
                  xleftmargin   = 0.8cm,
                  xrightmargin  = 0.8cm,
                ]
    sort3 := procedure(l) {
        [ x, y, z ] := l;
        if (x <= y) {
            if (y <= z) {
                return [ x, y, z ];
            } else if (x <= z) { 
                return [ x, z, y ];
            } else {
                return [ z, x, y ];
            }
        } else if (z <= y) { 
            return [z, y, x];
        } else if (x <= z) { 
            return [ y, x, z ];
        } else {
            return [ y, z, x ];
        }
    };
\end{Verbatim}
\vspace*{-0.3cm}
\caption{A function to sort a list of three elements.}
\label{fig:sort3.stlx}
\end{figure}

Figure \ref{fig:sort3switch.stlx} on page \pageref{fig:sort3switch.stlx} 
shows an equivalent program that uses a \texttt{switch} statement instead of an
\texttt{if-then-else} statement to sort a list of three elements.  Although this implementation is
easier to understand, it is less efficient than the 
previous version.  The reason is that some of the tests are redundant.  This is most obvious for the
last case in line 9 since at the time when control arrives in line 9 it is already known that $z$ must be
less or equal than $y$ and that, furthermore, $y$ must be less or equal than $x$, since all other
cases are already covered.

\begin{figure}[!ht]
\centering
\begin{Verbatim}[ frame         = lines, 
                  framesep      = 0.3cm, 
                  firstnumber   = 1,
                  labelposition = bottomline,
                  numbers       = left,
                  numbersep     = -0.2cm,
                  xleftmargin   = 0.8cm,
                  xrightmargin  = 0.8cm,
                ]
    sort3 := procedure(l) {
        [ x, y, z ] := l;
        switch {
            case x <= y && y <= z: return [ x, y, z ];
            case x <= z && z <= y: return [ x, z, y ];
            case y <= x && x <= z: return [ y, x, z ];
            case y <= z && z <= x: return [ y, z, x ];
            case z <= x && x <= y: return [ z, x, y ];
            case z <= y && y <= x: return [ z, y, x ];
            default: print("Impossible error occurred!");
        }
    };
\end{Verbatim}
\vspace*{-0.3cm}
\caption{Sorting a list of 3 elements using a \texttt{switch} statement.}
\label{fig:sort3switch.stlx}
\end{figure}
\noindent
The grammar rule describing the syntax of \texttt{switch} statements is as follows:
\\[0.2cm]
\hspace*{1.3cm}
$\textsl{stmnt} \;\rightarrow\; \quoted{switch} \quoted{\{} \textsl{caseList} \quoted{\}}$
\\[0.2cm]
where the syntactical variable \textsl{caseList} is defined via the rule:
\\[0.2cm]
\hspace*{1.3cm}
$\textsl{caseList} \;\rightarrow\; (\squoted{case}\; \textsl{boolExpr} \quoted{:} \textsl{block})^* 
 (\squoted{default} \quoted{:} \textsl{block})?$
\\[0.2cm]
Here, \textsl{boolExpr} is a Boolean expression and \textsl{block} represents a sequence of statements.

In contrast to the programming languages \texttt{C} and \textsl{Java}, the \texttt{switch}
statement in \setlx\ doesn't have a fall through.  Therefore, we don't need a \texttt{break} statement in the
block of statements following a case condition.  
There are two other important distinction between the \texttt{switch} statement in
\textsl{Java} and the \texttt{switch} statement in \setlx:  
\begin{enumerate}
\item In \textsc{SetlX}, the keyword switch is not followed by a value and
\item the conditions following the keyword \texttt{case} have to be Boolean values.  
\end{enumerate}
There is a another type of switching
statements that is much more powerful than the \texttt{switch} statement.  This is called
\emph{matching} and will be discussed in the next section.

\section{Matching}
One of the most powerful branching construct is \emph{matching}.  Although the syntax for the
matching statement is always the same, there are really four different variants of matching for each of the
data types that support matching.  Matching has been implemented for the data types
 \emph{strings}, \emph{lists}, \emph{sets}, and \emph{terms}.  We discuss \emph{string matching} first.

\subsection{String Matching}
Many algorithms that deal with a given string $s$ have to deal with two cases:  Either the string $s$ is
empty or it is nonempty and has to be split into its first character $c$ and the remaining characters $r$,
that is we have 
\\[0.2cm]
\hspace*{1.3cm}
$s = c + r$ \quad where $c = s[1]$ and $r = s[2..]$.
\\[0.2cm]
In order to facilitate algorithms that have to make this kind of case distinction, \setlx\ provides the
\texttt{match} statement.  Consider the function\footnote{
There is also a predefined version of the function \texttt{reverse} which does exactly the
same thing as the function in Figure \ref{fig:reverse.stlx}.  However, once we define the
function \texttt{reverse} as in Figure \ref{fig:reverse.stlx}, the predefined function
gets overwritten and is no longer accessible.}
 \texttt{reverse} shown in Figure \ref{fig:reverse.stlx}.
This function reverses its input argument, so the expression
\\[0.2cm]
\hspace*{1.3cm}
\texttt{reverse("abc")}
\\[0.2cm]
yields the result \texttt{"cba"}.  In order to reverse a string $s$, the function has to deal with two
cases:
\begin{enumerate}
\item The string $s$ is empty.  In this case, we can just return the string $s$ as it is.
      This case is dealt with in line 3.  There, we have used the pattern ``\texttt{[]}'' to match
      the empty string.  Instead, we could have used the empty string itself.  Using the pattern
      ``\texttt{[]}'' will prove beneficial when dealing with lists because it turns out that the
      function
      \texttt{reverse} as given in Figure \ref{fig:reverse.stlx} can also be used to reverse a list.
\item If the string $s$ is not empty, then it can be split up into a first character $c$ and the remaining
      characters $r$.  In this case, we reverse the string $r$ and append the character $c$ to the end of
      this string.  This case is dealt with in line 4.
\end{enumerate}
The \texttt{match} statement in function \texttt{reverse}  has a default case in line 5 to deal with
those cases where $s$ is neither a string nor a list. 

\begin{figure}[!ht]
\centering
\begin{Verbatim}[ frame         = lines, 
                  framesep      = 0.3cm, 
                  firstnumber   = 1,
                  labelposition = bottomline,
                  numbers       = left,
                  numbersep     = -0.2cm,
                  xleftmargin   = 0.8cm,
                  xrightmargin  = 0.8cm,
                ]
    reverse := procedure(s) {
        match (s) {
            case []   : return s;
            case [c|r]: return reverse(r) + c;
            default   : abort("type error in reverse($s$)");
        }
    };
\end{Verbatim}
\vspace*{-0.3cm}
\caption{A function that reverses a string.}
\label{fig:reverse.stlx}
\end{figure}

The last example shows that the syntax for the match statement  is similar to the syntax for the switch
statement.  The main difference is that the keyword is now ``\texttt{match}'' instead of
``\texttt{switch}'' and that the cases no longer contain Boolean values but instead contain
\emph{patterns} that can be used 
\begin{itemize}
\item to check whether a string has a given form and 
\item to extract certain components (like the first character or everything but the first character).  
\end{itemize}
Basically, for strings the function \texttt{reverse} given above is interpreted as if it
had been written in the way shown in Figure \ref{fig:reverse-long.stlx} on page \pageref{fig:reverse-long.stlx}.
This example shows that the use of the \texttt{match} statement can make programs more compact while
increasing their legibility.

\begin{figure}[!ht]
\centering
\begin{Verbatim}[ frame         = lines, 
                  framesep      = 0.3cm, 
                  firstnumber   = 1,
                  labelposition = bottomline,
                  numbers       = left,
                  numbersep     = -0.2cm,
                  xleftmargin   = 0.8cm,
                  xrightmargin  = 0.8cm,
                ]
    reverse := procedure(s) {
        if (s == "") {
            return s;
        } else if (isString(s)) {
            c := s[1];
            r := s[2..];
            return reverse(r) + c;
        } else {
            abort("type error in reverse($s$)");
        }
    };
\end{Verbatim}
\vspace*{-0.3cm}
\caption{A function to reverse a string that does not use matching.}
\label{fig:reverse-long.stlx}
\end{figure}

To explore string matching further, consider a function \texttt{reversePairs} that interchanges all pairs
of characters, so for example we have
\\[0.2cm]
\hspace*{1.3cm}
\texttt{reversePairs("abcd") = "badc"} \quad and \quad
\texttt{reversePairs("abcde") = "badce"}. 
\\[0.2cm]
This function can be implemented as shown in Figure \ref{fig:reverse-pairs.stlx} on page
\pageref{fig:reverse-pairs.stlx}.  Notice that we can match the empty string with the pattern
``\texttt{[]}'' in line 3.  The pattern ``\texttt{[c]}'' matches a string consisting of a single
character $c$.  In line 5 the pattern \texttt{[a,b|r]} extracts the first two
characters from the string $s$ and binds them to the variables $a$ and $b$.  The rest of the string is
bound to $r$.

\begin{figure}[!ht]
\centering
\begin{Verbatim}[ frame         = lines, 
                  framesep      = 0.3cm, 
                  firstnumber   = 1,
                  labelposition = bottomline,
                  numbers       = left,
                  numbersep     = -0.2cm,
                  xleftmargin   = 0.8cm,
                  xrightmargin  = 0.8cm,
                ]
    reversePairs := procedure(s) {
        match (s) {
            case []     : return s;
            case [c]    : return c;
            case [a,b|r]: return b + a + reversePairs(r);
    
        }
    };
\end{Verbatim}
\vspace*{-0.3cm}
\caption{A function to exchange pairs of characters.}
\label{fig:reverse-pairs.stlx}
\end{figure}

\subsubsection{String Decomposition via Assignment}
Assignment can be used to decompose a string into its constituent characters.  For
example, if $s$ is defined as
\\[0.2cm]
\hspace*{1.3cm}
\texttt{s := }\verb|"abc";|
\\[0.2cm]
then after the assignment
\\[0.2cm]
\hspace*{1.3cm}
\texttt{[u, v, w] := s;}
\\[0.2cm]
the variables $u$, $v$, and $w$ have the values
\\[0.2cm]
\hspace*{1.3cm}
\texttt{$u$ =} \verb|"a"|, \quad
\texttt{$v$ =} \verb|"b"|, \quad and \quad
\texttt{$w$ =} \verb|"c"|. 
\\[0.2cm]
Therefore, for strings, list assignment can be seen as a lightweight alternative to matching.

\subsection{List Matching}
As strings can be regarded as lists of characters, the matching of lists is very similar to the matching of
strings.  The function \texttt{reverse} shown in Figure \ref{fig:reverse.stlx} can also be used to
reverse a list.
 If the argument $s$ of \texttt{reverse} is a list instead of a string, we match the empty list with
 the pattern ``\texttt{[]}'', while the pattern ``\texttt{[c|r]}'' matches a non-empty list:  
The variable $c$  matches the first element  of the list, while the variable $r$ matches the remaining elements.

List assignment is another way to decompose a list that is akin to matching.  If the list 
$l$ is defined via
\\[0.2cm]
\hspace*{1.3cm}
\texttt{l := [1..3];}
\\[0.2cm]
then after the assignment
\\[0.2cm]
\hspace*{1.3cm}
\texttt{[x, y, z] := l;}
\\[0.2cm]
the variables $x$, $y$, and $y$ have the values $x = 1$, $y = 2$, and $z = 3$.

\subsection{Set Matching}
As sets are quite similar to lists, the matching of sets is closely related to the matching of
lists.  Figure \ref{fig:set-sort.stlx}  shows the function \texttt{setSort} that takes a
set of numbers as its  argument and returns a sorted list containing the numbers appearing
in the set.  In the \texttt{match} statement, we match the empty set with the pattern
``\texttt{\{\}}'', while the pattern ``\texttt{\{x|r\}}'' matches a non-empty set:  
The variable $x$ matches the first element of the set, while the variable $r$ matches the
set of all the remaining elements.  


\begin{figure}[!ht]
\centering
\begin{Verbatim}[ frame         = lines, 
                  framesep      = 0.3cm, 
                  firstnumber   = 1,
                  labelposition = bottomline,
                  numbers       = left,
                  numbersep     = -0.2cm,
                  xleftmargin   = 0.8cm,
                  xrightmargin  = 0.8cm,
                ]
    setSort := procedure(s) {
        match (s) {
            case {}   : return [];
            case {x|r}: return [x] + setSort(r);
        }
    };
\end{Verbatim}
\vspace*{-0.3cm}
\caption{A function to sort a set of numbers.}
\label{fig:set-sort.stlx}
\end{figure}

Of course, sorting a set into a list is trivial, as a
set in \setlx\ is represented by an ordered binary tree and therefore is already sorted.
For this reason, we could also transform a set $s$ into a sorted list by using the expression
\\[0.2cm]
\hspace*{1.3cm}
\texttt{[] + s}.
\\[0.2cm]
In general, for a list $l$ and a set $s$ the expression \texttt{$l$ + $s$} creates a new
list containing all elements of $l$.  Then, the elements of $s$ are appended to this
list.  The expression \texttt{$s$ + $l$} creates a new set containing the elements of
$s$.  Then, the elements of $l$ are inserted into this set.


\subsection{Term Matching}
The most elaborate form of matching is the matching of terms.  This kind of matching is similar to
the kind of matching provided in the programming languages 
\href{http://en.wikipedia.org/wiki/Prolog}{\textsl{Prolog}} and 
\href{http://en.wikipedia.org/wiki/ML_(programming_language)}{\textsc{Ml}} \cite{milner:90}.
Figure \ref{fig:binary-tree.stlx} shows an implementation of the function \texttt{insert} to insert
a number into an ordered binary tree.  This implementation uses
term matching instead of the functions ``\texttt{fct}'' and ``\texttt{args}'' that had been used in
the previous implementation shown in Figure \ref{fig:binary-tree-no-matching.stlx} on page
\pageref{fig:binary-tree-no-matching.stlx}.  In line 3 of Figure \ref{fig:binary-tree.stlx}, the
\texttt{case} statement checks whether $m$ is the empty tree.  This is more straightforward than
testing that the functor of $m$ is ``\texttt{Nil}'', as it is done in line 3 of Figure
\ref{fig:binary-tree-no-matching.stlx}.  However, the real benefit of matching shows in line 5 of
Figure \ref{fig:binary-tree.stlx} since the case statement in this line does not only check whether
the functor of $m$ is ``\texttt{Node}'' but also assigns the variables \texttt{k2}, \texttt{l}, and
\texttt{r} to the respective subterms of $m$.  Compare this with line 5 and line 6 of Figure
\ref{fig:binary-tree-no-matching.stlx} where we had to use a separate statement in line 6 to extract
the arguments of $m$.


\begin{figure}[!ht]
\centering
\begin{Verbatim}[ frame         = lines, 
                  framesep      = 0.3cm, 
                  firstnumber   = 1,
                  labelposition = bottomline,
                  numbers       = left,
                  numbersep     = -0.2cm,
                  xleftmargin   = 0.8cm,
                  xrightmargin  = 0.8cm,
                ]
    insert := procedure(m, k1) {
        match (m) {
            case Nil() : 
                 return Node(k1, Nil(), Nil());
            case Node(k2, l, r): 
                 if (k1 == k2) {
                     return Node(k1, l, r);
                 } else if (compare(k1, k2) < 0) { 
                     return Node(k2, insert(l, k1), r);
                 } else {
                     return Node(k2, l, insert(r, k1));
                 }
            default: abort("Error in insert($m$, $k1$, $v1$)");
        }
    };
\end{Verbatim}
\vspace*{-0.3cm}
\caption{Inserting an element into a binary tree using matching.}
\label{fig:binary-tree.stlx}
\end{figure}

\begin{figure}[!ht]
\centering
\begin{Verbatim}[ frame         = lines, 
                  framesep      = 0.3cm, 
                  firstnumber   = 1,
                  labelposition = bottomline,
                  numbers       = left,
                  numbersep     = -0.2cm,
                  xleftmargin   = 0.8cm,
                  xrightmargin  = 0.8cm,
                ]
    diff := procedure(t, x) {
        match (t) {
            case t1 + t2 :
                return diff(t1, x) + diff(t2, x);
            case t1 - t2 :
                return diff(t1, x) - diff(t2, x);
            case t1 * t2 :
                return diff(t1, x) * t2 + t1 * diff(t2, x);
            case t1 / t2 :
                return ( diff(t1, x) * t2 - t1 * diff(t2, x) ) / t2 * t2;
            case f ** g :
                return diff( @exp(g * @ln(f)), x);
            case ln(a) :
                return diff(a, x) / a;
            case exp(a) :
                return diff(a, x) * @exp(a);
            case ^variable(x) : // x is defined above as second argument
                return 1;
            case ^variable(y) : // y is undefined, matches any other variable
                return 0;
            case n | isNumber(n):   
                return 0;  
        }
    };
\end{Verbatim}
\vspace*{-0.3cm}
\caption{A function to perform symbolic differentiation.}
\label{fig:diff.stlx}
\end{figure}


Let us discuss a more complex example of matching.  The function \texttt{diff} shown in Figure
\ref{fig:diff.stlx} on page \pageref{fig:diff.stlx} is supposed to be called with two arguments:
\begin{enumerate}
\item The first argument $t$ is a term that is interpreted as an arithmetic expression.
\item The second argument $x$ is a string that is interpreted as  the name of a variable.
\end{enumerate}
The function \texttt{diff} interprets the term $t$ as a mathematical function and differentiates this
function with respect to the variable $x$.  For example, in order to compute the derivative of the function
\\[0.2cm]
\hspace*{1.3cm}
$x \mapsto x^x$
\\[0.2cm]
we can invoke \texttt{diff} as follows:
\\[0.2cm]
\hspace*{1.3cm}
\texttt{diff(parse("x ** x"), "x");}
\\[0.2cm]
Here the function \texttt{parse} transforms the string ``\texttt{x ** x}'' into a term.  The form of
this term will be discussed in more detail later.  For the moment, let us focus on the
\texttt{match} statement in Figure
\ref{fig:diff.stlx}.  Consider line 3: If the term that is to be differentiated has the form
\texttt{t1 + t2}, then both \texttt{t1} and \texttt{t2} have to be differentiated separately and the
resulting terms have to be added.  For a more interesting example, consider line 8.  This line
implements the product rule:
\\[0.2cm]
\hspace*{1.3cm}
$\frac{d\;}{dx} \bigl(t_1 \cdot t_2\bigr) = \frac{d\, t_1}{dx} \cdot t_2 + t_1 \cdot \frac{d\,t_2}{dx}$.
\\[0.2cm]
Note how the pattern 
\\[0.2cm]
\hspace*{1.3cm}
\texttt{t1 * t2}
\\[0.2cm]
in line 7 extracts the two factors from a term that is a product.  Further, note that in line 12 and line 16
we had to prefix the function symbols ``\texttt{exp}'' and ``\texttt{ln}'' with the character
``\texttt{\symbol{64}}'' in order to convert these function symbols into functors.

In order to understand this example in Figure \ref{fig:diff.stlx} in more
detail, we have to discuss how the function \texttt{parse} converts a string into a term.  The
function \texttt{parse} needs to represent all operator symbols and it also needs to represent
variables.  A variable of the form \texttt{\symbol{34}x\symbol{34}} is parsed as a
term of the form
\\[0.2cm]
\hspace*{1.3cm}
\texttt{\symbol{94}variable(\symbol{34}x\symbol{34})}.
\\[0.2cm]
This should explain the patterns used in line 19 and line 21 of Figure \ref{fig:diff.stlx}.
In order to inspect the internal representation of a term, we can use the function
``\texttt{canonical}''.  For example, the expression
\\[0.2cm]
\hspace*{1.3cm}
\texttt{canonical(parse(\symbol{34}x ** x\symbol{34}))}
\\[0.2cm]
yields the result
\\[0.2cm]
\hspace*{1.3cm}
\texttt{\symbol{94}power(\symbol{94}variable(\symbol{34}x\symbol{34}), \symbol{94}variable(\symbol{34}x\symbol{34}))}.
\\[0.2cm]
This shows that the functor ``\texttt{\symbol{94}power}'' is the internal representation of the
power operator ``\texttt{**}''.  The internal representation of 
``\texttt{+}'' is ``\texttt{\symbol{94}sum}'',
``\texttt{-}'' is represented as ``\texttt{\symbol{94}difference}'',
``\texttt{*}'' is represented as ``\texttt{\symbol{94}product}'', and
``\texttt{/}'' is represented as ``\texttt{\symbol{94}quotient}''.

Note that the example makes extensive use of the fact that terms are \emph{viral} when used with
the arithmetic operators 
``\texttt{+}'', ``\texttt{-}'', ``\texttt{*}'', ``\texttt{/}'', ``\texttt{\symbol{92}}'', and
``\texttt{\%}'':  
If one operand of these operators is a term, the operator automatically yields a term.
For example, the expression 
\\[0.2cm]
\hspace*{1.3cm}
\texttt{parse("x") + 2}
\\[0.2cm]
yields the term
\\[0.2cm]
\hspace*{1.3cm}
\texttt{\symbol{94}sum(\symbol{94}variable(\symbol{34}x\symbol{34}), 2)}.
\\[0.2cm]
Note also that terms are not viral inside function symbols like ``\texttt{exp}''.  Therefore, this
function symbol has to be prefixed by the operator ``\texttt{\symbol{64}}'' to turn it
into a functor.

Line 21 shows how a condition can be attached to a pattern:  The pattern 
\\[0.2cm]
\hspace*{1.3cm}
\texttt{case n:}
\\[0.2cm]
would match anything.  However, we want to match only numbers here.  Therefore, we have used
the pattern
\\[0.2cm]
\hspace*{1.3cm}
\texttt{case n | isNumber(n):}
\\[0.2cm]
in order to ensure that \texttt{n} is indeed a number.

\subsection{Term Decomposition via List Assignment}
Similar to strings, terms can be decomposed via list assignment.  For example, after the assignment
\\[0.2cm]
\hspace*{1.3cm}
\texttt{[x,y,z] := F(1, G(2), \{2,3\});}
\\[0.2cm]
the variables $x$, $y$, and $z$ have the values
\\[0.2cm]
\hspace*{1.3cm}
$x = 1$, \quad $y = \mathtt{G(2)}$, \quad and \quad $z = \{2,3\}$.
\\[0.2cm]
Of course, the function \texttt{args} achieves a similar effect.  We have that
\\[0.2cm]
\hspace*{1.3cm}
\texttt{args(F(1, G(2), \{2,3\})) = [1, G(2), \{2, 3\}]}.

\section{Loops}
\setlx\ offers three different kinds of loops: \texttt{for} loops, \texttt{while} loops, and
\texttt{do-while} loops.  The \texttt{while} loops are the most general loops.  Therefore, we discuss them first.  

\subsection{\texttt{while} Loops}
The syntax and semantics of \texttt{while} loops in \setlx\ is really the same as in the programming
language \texttt{C}.  To demonstrate a \texttt{while} loop,
let us implement a function testing the 
\href{http://en.wikipedia.org/wiki/Collatz_conjecture}{\emph{Collatz conjecture}}:  Define the function
\\[0.2cm]
\hspace*{1.3cm}
$f: \mathbb{N} \rightarrow \mathbb{N}$
\\[0.2cm]
recursively as follows:
\begin{enumerate}
\item $f(n) := 1$ \hspace*{2.13cm} if $n \leq 1$,
\item $f(n) := \left\{
       \begin{array}[c]{ll}
         f(n/2)           & \mbox{if $n \,\texttt{\symbol{37}}\, 2 = 0$;} \\[0.2cm]  
         f(3 \cdot n + 1) & \mbox{otherwise.} 
       \end{array}
       \right.
      $ 
\end{enumerate}
The Collatz conjecture claims that $f(n) = 1$  for all $n \in \mathbb{N}$. 
If we assume the Collatz conjecture is true, then $f$ is well-defined.
Otherwise, if the Collatz conjecture is not true, for certain values of $n$ the function $f(n)$ is undefined.
Figure \ref{fig:ulam.stlx} shows an implementation of the function $f$ in \setlx.  

\begin{figure}[!ht]
\centering
\begin{Verbatim}[ frame         = lines, 
                  framesep      = 0.3cm, 
                  firstnumber   = 1,
                  labelposition = bottomline,
                  numbers       = left,
                  numbersep     = -0.2cm,
                  xleftmargin   = 0.8cm,
                  xrightmargin  = 0.8cm,
                ]
    f := procedure(n) {
        if (n == 0) {
            return 1;   
        }
        while (n != 1) {
            if (n % 2 == 0) {
                n /= 2;
            } else {
                n := 3 * n + 1;
            }
        }
        return n;
    };
\end{Verbatim}
\vspace*{-0.3cm}
\caption{A program to test the Collatz conjecture.}
\label{fig:ulam.stlx}
\end{figure}

The function $f$ is implemented via a \texttt{while} loop.  This loop runs as long as $n$
is different from the number one.  Therefore, if the Collatz conjecture is true, the \texttt{while}
loop will eventually terminate for all values of $n$.

The syntax of a \texttt{while} loop is given by the following
grammar rule:
\\[0.2cm]
\hspace*{1.3cm}
$\textsl{statement} \;\rightarrow\; \quoted{while} \quoted{(} \textsl{boolExpr} \quoted{)}
 \quoted{\{}  \textsl{block} \quoted{\}} 
$.
\\[0.2cm]
Here, \textsl{boolExpr} is a Boolean expression returning either \texttt{true} or
\texttt{false}.  This condition is called the \emph{guard} of the \texttt{while} loop.
The syntactical variable \textsl{block} denotes a sequence of statements.  Note that, 
in contrast to the programming languages \texttt{C} and \textsl{Java}, the
block of statements always has to be enclosed in curly braces, even if it consists only of a
single statement.  The semantics of a \texttt{while} loop is the same as in \texttt{C}:
The loop is executed as long as the guard is true.  In order to abort an iteration
prematurely, \setlx\ provides the command \texttt{continue}.  This command aborts the
current iteration of the loop and proceeds with the next iteration.  In order to abort the
loop itself, the command \texttt{break} can be used.  Figure \ref{fig:break-and-continue.stlx} shows
a function that uses both a \texttt{break} statement and a \texttt{continue} statement.  This
function will print the number 1.  Then, when $n$ is incremented to 2, the \texttt{continue}
statement in line 6 is executed so that the number 2 is not printed.  In the next iteration of the
loop, the number $n$ is incremented to 3 and printed.  In the final iteration of the loop, $n$ is
incremented to 4 and the \texttt{break} statement in line 9 terminates the loop.


\begin{figure}[!ht]
\centering
\begin{Verbatim}[ frame         = lines, 
                  framesep      = 0.3cm, 
                  firstnumber   = 1,
                  labelposition = bottomline,
                  numbers       = left,
                  numbersep     = -0.2cm,
                  xleftmargin   = 0.8cm,
                  xrightmargin  = 0.8cm,
                ]
    testBreakAndContinue := procedure() {
        n := 0;
        while (n < 10) {
            n := n + 1;
            if (n == 2) {
                continue;
            }
            if (n == 4) {
                break;
            }
            print(n);
        }
    };
\end{Verbatim}
\vspace*{-0.3cm}
\caption{This function demonstrates the semantics of \texttt{break} and \texttt{continue}.}
\label{fig:break-and-continue.stlx}
\end{figure}


\subsection{\texttt{do-while} Loops}
Similar to the language \texttt{C}, \setlx\ supports the \texttt{do-while} loop.  The difference
between a \texttt{do-while} loop and an ordinary \texttt{while} loop is that 
sometimes the body of a loop needs to execute at least once, regardless of the condition
controlling the loop.  For example, imagine you want to implement the following guessing game:  The computer
thinks of a natural number between 0 and 100 inclusive and the player has to guess it.  Every time the
player enters some number, the computer informs the player whether the number was too big, too
small, or whether the player has correctly guessed the secret number.  In this guessing game, the
player always has to enter at least one number.  Therefore, the most natural way to implement this
game is to use a \texttt{do-while} loop.  Figure \ref{fig:guessNumber.stlx} shows an implementation
of the guessing game.  The implementation works as follows:
\begin{enumerate}
\item In line 2, the secret number is generated as a random number.  This is the number that has to
      be guessed by the player.
\item The variable \texttt{count} is used to count the number of guessing attempts.
      This variable is initialized in line 3.
\item Since the user has to enter at least one number, we use a \texttt{do-while} loop that loops as
      long as the user has not yet guessed the secret number.

      In line 6, the user is asked to guess the secret number.  If this number is either too small
      or too big, an appropriate message is printed.  The loop terminates in line 14 if the number
      that has been guessed is identical to the secret number.
\end{enumerate}


\begin{figure}[!ht]
\centering
\begin{Verbatim}[ frame         = lines, 
                  framesep      = 0.3cm, 
                  firstnumber   = 1,
                  labelposition = bottomline,
                  numbers       = left,
                  numbersep     = -0.2cm,
                  xleftmargin   = 0.8cm,
                  xrightmargin  = 0.8cm,
                ]
    guessNumber := procedure() {
        secret := rnd(100);
        count  := 0;
        do {
            count += 1;
            x := read("input a number between 0 and 100 inclusively: ");
            if (x < secret) {
                print("sorry, too small");
            } else if (x > secret) {
                print("sorry, too big");
            } else {
                print("correct!");
            }
        } while (x != secret);
        print("number of guesses: $count$");
    };
\end{Verbatim}
\vspace*{-0.3cm}
\caption{Implementing the guessing game in \setlx.}
\label{fig:guessNumber.stlx}
\end{figure}

\noindent
The syntax of a \texttt{do-while} loop is given by the following grammar rule:
\\[0.2cm]
\hspace*{1.3cm}
$\textsl{statement} \;\rightarrow\; \quoted{do} \quoted{\{}  \textsl{block} \quoted{\}} \quoted{while}
  \quoted{(} \textsl{boolExpr} \quoted{)} \quoted{;} 
$.
\\[0.2cm]
Here, \textsl{boolExpr} is a Boolean expression returning either \texttt{true} or
\texttt{false}.  This expression is called the \emph{guard} of the \texttt{do-while} loop.
The syntactical variable \textsl{block} denotes a sequence of statements.  Note that, 
in contrast to the programming languages \texttt{C} and \textsl{Java}, the
block of statements always has to be enclosed in curly braces, even if it consists only of a
single statement.  Furthermore, note that a \texttt{do-while} loop has to be terminated with a semicolon.

The semantics of a \texttt{do-while} loop is the same as in \texttt{C}:
The body of the loop is executed once.  Then, if the guard is false, execution terminates.  Otherwise,
the body is executed again and again as long as the guard is true.  In order
to abort an iteration of the loop prematurely, the commands \texttt{continue} and
\texttt{break} can be used.  These commands work in the same way as in a \texttt{while} loop.

\subsection{\texttt{for} Loops}
In order to perform a list of commands a predefined number of times, a \texttt{for} loop should  be
used.  Figure \ref{fig:multiplication-table.stlx} on page \pageref{fig:multiplication-table.stlx}
shows some \setlx\ code that prints a multiplication table.  The output of this program is
shown in Figure \ref{fig:multiplication-table} on page \pageref{fig:multiplication-table}.

In the program in Figure \ref{fig:multiplication-table.stlx}, the printing is done in the two nested
loops that start in line 8.  In line 8, the counting variable \texttt{i} iterates over all values
from $1$ to $10$.  Similarly, the counting variable \texttt{j} in line 9 iterates over the same values.
The product \texttt{i * j} is computed in line 10 and printed without a newline using the
function \texttt{nPrint}.
The function $\texttt{rightAdjust}(n)$ turns the number $n$ into a string by padding the
number with blanks from the left so that the resulting string always has a length of 4 
characters.

\begin{figure}[!ht]
\centering
\begin{Verbatim}[ frame         = lines, 
                  framesep      = 0.3cm, 
                  firstnumber   = 1,
                  labelposition = bottomline,
                  numbers       = left,
                  numbersep     = -0.2cm,
                  xleftmargin   = 0.8cm,
                  xrightmargin  = 0.8cm,
                ]
    rightAdjust := procedure(n) {
        switch {
            case n < 10 : return "   " + n;
            case n < 100: return  "  " + n;
            default:      return   " " + n;
        }
    };      
    for (i in [1 .. 10]) {
        for (j in [1 .. 10]) {
            nPrint(rightAdjust(i * j));
        }
        print();
    }
\end{Verbatim}
\vspace*{-0.3cm}
\caption{A simple program to generate a multiplication table.}
\label{fig:multiplication-table.stlx}
\end{figure}

\begin{figure}[!ht]
\centering
\begin{Verbatim}[ frame         = lines, 
                  framesep      = 0.3cm, 
                  firstnumber   = 1,
                  labelposition = bottomline,
                  numbers       = none,
                  numbersep     = -0.2cm,
                  xleftmargin   = 0.8cm,
                  xrightmargin  = 0.8cm,
                ]
       1   2   3   4   5   6   7   8   9  10
       2   4   6   8  10  12  14  16  18  20
       3   6   9  12  15  18  21  24  27  30
       4   8  12  16  20  24  28  32  36  40
       5  10  15  20  25  30  35  40  45  50
       6  12  18  24  30  36  42  48  54  60
       7  14  21  28  35  42  49  56  63  70
       8  16  24  32  40  48  56  64  72  80
       9  18  27  36  45  54  63  72  81  90
      10  20  30  40  50  60  70  80  90 100
\end{Verbatim}
\vspace*{-0.3cm}
\caption{Output of the program in Figure \ref{fig:multiplication-table.stlx}.}
\label{fig:multiplication-table}
\end{figure}

The general syntax of a \texttt{for} loop is given by the following \textsc{Ebnf} rule:
\\[0.2cm]
\hspace*{1.3cm}
$\textsl{statement} \rightarrow \quoted{for} \quoted{(} \textsl{iterator} \;(\squoted{,}\; \textsl{iterator})^* \quoted{)}
  \quoted{\{} \textsl{block} \;\squoted{\}}
$.
\\[0.2cm]
Here an iterator is either a \emph{simple iterator} or a \emph{tuple iterator}.  A
\emph{simple iterator} has the form
\\[0.2cm]
\hspace*{1.3cm}
$x \quoted{in} s$
\\[0.2cm]
where $s$ is either a set, a list, or a string and $x$ is the name of a variable.
This variable is bound to the elements of $s$ in turn.  For example,  the statement
\\[0.2cm]
\hspace*{1.3cm}
\texttt{for (x in [1..10]) \{ print(x); \}}
\\[0.2cm]
will print the numbers from 1 to 10.  If $s$ is a string, then the variable $x$ iterates over the
characters of $s$.  For example, the statement
\\[0.2cm]
\hspace*{1.3cm}
\texttt{for (c in \symbol{34}abc\symbol{34}) \{ print(c); \}}
\\[0.2cm]
prints the characters \texttt{\symbol{34}a\symbol{34}}, \texttt{\symbol{34}b\symbol{34}},
and \texttt{\symbol{34}c\symbol{34}}.  


The iterator of a \texttt{for} loop can also be a \emph{tuple iterator}.  The simplest form of
a tuple is
\\[0.2cm]
\hspace*{1.3cm}
 $[x_1, \cdots, x_n] \quoted{in} s$.
\\[0.2cm]
Here, $s$ must be either a set or a list that contains lists of length $n$ as elements. 
Figure \ref{fig:relational-product-for.stlx} on page \pageref{fig:relational-product-for.stlx}
shows a procedure that computes the relational product 
of two binary relations $r_1$ and $r_2$.  In set theory, the relational product $r_1 \circ r_2$
is defined as
\\[0.2cm]
\hspace*{1.3cm}
$r_1 \circ r_2 := \{ \pair(x,z) \mid \pair(x,y) \in r_1 \wedge \pair(y,z) \in r_2 \}$.
\\[0.2cm]
The \texttt{for} loop in line 3 iterates over the two relations \texttt{r1} and \texttt{r2}.
The following \texttt{if} statement selects those pairs of pairs of numbers such that the second
component of the first pair is identical to the first component of the second pair.

\begin{figure}[!ht]
\centering
\begin{Verbatim}[ frame         = lines, 
                  framesep      = 0.3cm, 
                  firstnumber   = 1,
                  labelposition = bottomline,
                  numbers       = left,
                  numbersep     = -0.2cm,
                  xleftmargin   = 0.8cm,
                  xrightmargin  = 0.8cm,
                ]
    product := procedure(r1, r2) {
        r := {};
        for ([x, y1] in r1, [y2, z] in r2) {
            if (y1 == y2) {
                r += { [x, z] };
            }
        }
        return r;
    };
\end{Verbatim}
\vspace*{-0.3cm}
\caption{A program to compute the relational product of two binary relations.}
\label{fig:relational-product-for.stlx}
\end{figure}

Of course, in \setlx\ the relational product can be computed more easily via set comprehension.
Figure \ref{fig:relational-product.stlx} on page \pageref{fig:relational-product.stlx} shows
an implementation that is based on set comprehension.  It is a bit shorter as the test in
line 4 of Figure \ref{fig:relational-product-for.stlx} is essentially integrated in the
definition of the set in line 2 of Figure \ref{fig:relational-product.stlx}.
In general, most occurrences of \texttt{for} loops can be replaced by equivalent set
definitions. Our experience shows that the resulting code will be both shorter and easier to understand.


\begin{figure}[!ht]
\centering
\begin{Verbatim}[ frame         = lines, 
                  framesep      = 0.3cm, 
                  firstnumber   = 1,
                  labelposition = bottomline,
                  numbers       = left,
                  numbersep     = -0.2cm,
                  xleftmargin   = 0.8cm,
                  xrightmargin  = 0.8cm,
                ]
    product := procedure(r1, r2) {
        return { [x, z] : [x, y] in r1, [y, z] in r2 };
    };
\end{Verbatim}
\vspace*{-0.3cm}
\caption{Computing the relational product via set comprehension.}
\label{fig:relational-product.stlx}
\end{figure}

Iterators can be even more complex, since the tuples can be nested, so something like
\\[0.2cm]
\hspace*{1.3cm}
\texttt{for ([[x,y],z] in s) \{ $\cdots$ \}}
\\[0.2cm]
is possible.  However, as this feature is rarely needed, we won't discuss it in more detail.


A \texttt{for} loop creates a 
\href{http://en.wikipedia.org/wiki/Variable_scope}{\emph{local scope} }
for the iteration variable.  This means that changes to the variable \texttt{x} that occur in the
\texttt{for} loop are not visible outside of the \texttt{for} loop.
Therefore, the last
line of the program shown in Figure \ref{fig:scope-for-loop.stlx} prints the message
\\[0.2cm]
\hspace*{1.3cm}
\texttt{x = 1.}

\begin{figure}[!ht]
\centering
\begin{Verbatim}[ frame         = lines, 
                  framesep      = 0.3cm, 
                  firstnumber   = 1,
                  labelposition = bottomline,
                  numbers       = left,
                  numbersep     = -0.2cm,
                  xleftmargin   = 0.8cm,
                  xrightmargin  = 0.8cm,
                ]
    x := 1;
    for (x in "abc") {
        print(x);
    }
    print("x = $x$");
\end{Verbatim}
\vspace*{-0.3cm}
\caption{A program illustrating the scope of a \texttt{for} loop.}
\label{fig:scope-for-loop.stlx}
\end{figure}




%%% Local Variables: 
%%% mode: latex
%%% TeX-master: "tutorial"
%%% End: 
