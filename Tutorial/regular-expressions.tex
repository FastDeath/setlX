\chapter{Regular Expressions \label{chapter:regular-expressions}}
\emph{Regular expressions} are a very powerful tool when processing strings.
Therefore, most modern programming languages support them.  \setlx\ is no exception.  
As \setlx\ is based on \textsl{Java}, the syntax of the regular expressions supported by
\setlx\ is the same as the syntax of regular expressions in \textsl{Java}.  Therefore, it
is not necessary to describe the syntax of regular expressions in this document.  The
documentation of the \textsl{Java} class \texttt{java.util.regex.Pattern} which can
be found at
\\[0.2cm]
\hspace*{1.3cm}
\href{http://docs.oracle.com/javase/7/docs/api/java/util/regex/Pattern.html}{\texttt{http://docs.oracle.com/javase/7/docs/api/java/util/regex/Pattern.html}}
\\[0.2cm]
contains a concise description of the syntax and semantics of regular expressions.
For a more general discussion of the use of regular expression, the book of Friedl
\cite{friedl:2006} is an excellent choice.
Therefore, we will confine ourselves to show how regular expressions can be used in
\setlx.  In \setlx, there are two control structures that make use of regular expressions.
The first is the \texttt{match} statement and the second is the \texttt{scan} statement.

\section{Using Regular Expressions in a \texttt{match} Statement}
Instead of the keyword ``\texttt{case}'', a branch in a \texttt{match} statement can begin
with the keyword ``\texttt{regex}''.  Figure \ref{fig:regexp.stlx} shows the definition of
a function named \texttt{classify} that can takes a string and tries to classify this
string as either a word or a number.  


\begin{figure}[!ht]
\centering
\begin{Verbatim}[ frame         = lines, 
                  framesep      = 0.3cm, 
                  firstnumber   = 1,
                  labelposition = bottomline,
                  numbers       = left,
                  numbersep     = -0.2cm,
                  xleftmargin   = 0.8cm,
                  xrightmargin  = 0.8cm,
                ]
    classify := procedure(s) {
        match (s) {
            regex '0|[1-9][0-9]*': print("found an integer");
            regex '[a-zA-Z]+'    : print("found a word");
            regex '\s+'          : // skip white space
            default              : print("unkown: $s$");
        }
    };
\end{Verbatim}
\vspace*{-0.3cm}
\caption{A simple function to recognize numbers and words.}
\label{fig:regexp.stlx}
\end{figure}

Note that we have specified the regular expressions
using literal strings, i.e.~strings enclosed in single quote characters. This is necessary
in line 5, since the regular expression \\[0.2cm]
\hspace*{1.3cm}
\verb|'\s+'|
\\[0.2cm]
contains a backslash character.  If we had used double quotes, it would have been
necessary to escape the backslash character with another backslash and we would have had
to write
\\[0.2cm]
\hspace*{1.3cm}
\verb|"\\s+"|
\\[0.2cm]
instead.  Invoking the function \texttt{classify} as 
\\[0.2cm]
\hspace*{1.3cm}
\texttt{classify("123");}
\\[0.2cm]
 prints the message ``\texttt{found an integer}'', while invoking the function as
\\[0.2cm]
\hspace*{1.3cm}
\texttt{classify("Hugo");}
\\[0.2cm]
prints the message `\texttt{found a word}''.  Finally, calling
\\[0.2cm]
\hspace*{1.3cm}
\texttt{classify("0123");}
\\[0.2cm]
prints the answer ``\texttt{unkown: 0123}''.  

\subsection{Extracting Substrings}
Often,  strings are structured and the task is to extract substrings corresponding to
certain parts of a string.  This can be done using regular expressions.  For 
example, consider a phone number in the format
\\[0.2cm]
\hspace*{1.3cm}
\texttt{+49-711-6673-4504}.
\\[0.2cm]
Here, the substring ``\texttt{49}'' is the country code, the substring ``\texttt{711}'' is
the area code, the substring ``\texttt{6673}'' is the company code, and finally
``\texttt{4504}'' is the extension.  The regular expression 
\\[0.2cm]
\hspace*{1.3cm}
\verb|\+([0-9]+)-([0-9]+)-([0-9]+)-([0-9]+)|
\\[0.2cm]
specifies this format and the different blocks of parentheses correspond to the diffferent
codes. If  a phone number is given and the task is to
extract, say, the country code and the area code, then this can be achieved with the 
\setlx\ function shown in Figure \ref{fig:extract-phone-code.stlx}.

\begin{figure}[!ht]
\centering
\begin{Verbatim}[ frame         = lines, 
                  framesep      = 0.3cm, 
                  firstnumber   = 1,
                  labelposition = bottomline,
                  numbers       = left,
                  numbersep     = -0.2cm,
                  xleftmargin   = 0.3cm,
                  xrightmargin  = 0.3cm,
                ]
    extractCountryArea := procedure(phone) {
        match (phone) {
            regex '\+([0-9]+)-([0-9]+)-([0-9]+)-([0-9]+)' as [_, c, a, _, _]:
                return [c, a];
            default: abort("The string $phone$ is not a phone number!");
        }
    };
\end{Verbatim}
\vspace*{-0.3cm}
\caption{A function to extract the country and area code of a phone number.}
\label{fig:extract-phone-code.stlx}
\end{figure}

\noindent
Here, the regular expression to recognize phone numbers has several parts that are
enclosed in parentheses.  These parts are collected in a list of the form
\\[0.2cm]
\hspace*{1.3cm}
$[s, p_1, \cdots, p_n]$.
\\[0.2cm]
The first element $s$ of this list is the string that matched the regular expression.  The
remaning elements $p_i$ correspond to the different parts of the regular expression: $p_1$
corresponds to the first group of parentheses, $p_2$ corresponds to the second group, and
in general $p_i$ corresponds to the $i$-th group.  In line 3 this list is the matched
against the pattern
\\[0.2cm]
\hspace*{1.3cm}
\texttt{[\_, c, a, \_, \_]}.
\\[0.2cm]
Therefore, if the match is successful, the variable \texttt{c} will contain the country
code and the variable \texttt{a} is instantiated with the country code.  The groups of
regular expressions that are not needed are matched against the anonymous variable
``\texttt{\_}''. 

If there a regular expression contains  nested groups of parentheses, then the order of
the groups is determined by the left parenthesis of a group.  For example, the regular
expression 
\\[0.2cm]
\hspace*{1.3cm}
\texttt{'((a+)(b*)c?)d'}
\\[0.2cm]
contains three groups:
\begin{enumerate}
\item the first group is \texttt{'((a+)(b*)c?)'},
\item the second group is \texttt{'(a+)'}, and
\item the third group is \texttt{'(b*)'}.
\end{enumerate}

\subsection{Testing Regular Expressions}
In real life applications, regular expressions can get quite involved and difficult to
comprehend.  The function \texttt{testRegexp} shown in Figure \ref{fig:test-regexp.stlx}
can be used to test a given regular expression:  The function takes a regular expression
\texttt{re} as its first argument, while the second argument is a string \texttt{s}.  The
function tests whether the string \texttt{s} matches the regular expression \texttt{re}.
If this is the case, the function \texttt{testRegexp} returns a list that contains all the
substrings corresponding to the different parenthesized groups of the regular expression
\texttt{re}.   For example,  invoking this function as
\\[0.2cm]
\hspace*{1.3cm}
\texttt{testRegexp('(a*)(a+)b', "aaab");}
\\[0.2cm]
yields the result
\\[0.2cm]
\hspace*{1.3cm}
\texttt{["aaab", "aa", "a"]}.
\\[0.2cm]
Here, the first element of the list is the string that was matched by the regular
expression, the second element \texttt{"aa"} corresponds to the regular subexpression
\texttt{'(a*)'}, and the last element \texttt{"a"} corresponds to the regular
subexpression \texttt{'(a+)'}.

\begin{figure}[!ht]
\centering
\begin{Verbatim}[ frame         = lines, 
                  framesep      = 0.3cm, 
                  firstnumber   = 1,
                  labelposition = bottomline,
                  numbers       = left,
                  numbersep     = -0.2cm,
                  xleftmargin   = 0.8cm,
                  xrightmargin  = 0.8cm,
                ]
    testRegexp := procedure(re, s) {
        match (s) {
            regex re as l: return l;
            default : print("no match!");
        }
    };
\end{Verbatim}
\vspace*{-0.3cm}
\caption{Testing regular expressions.}
\label{fig:test-regexp.stlx}
\end{figure}

\subsection{Extracting Comments from a File}
In this section we present an example of the  \texttt{match} statement in action.  The
function \texttt{printComments} shown in Figure \ref{fig:find-comments.stlx} attempts to
extract all those \texttt{C}-style comments from a file that are contained in a single
line.  The regular expression \verb|'\s*(//.*)'| in line 5 matches comments starting
with ``\texttt{//}'', while the regular expression
\\[0.2cm]
\hspace*{1.3cm}
\verb"'\s*(\/\*([^*]|\*+[^*/])*\*+\/)\s*'"
\\[0.2cm]
in line 6 matches comments that start with the string ``\texttt{/*}'' and end with the
string ``\texttt{*/}''.  This regular expression is quite difficult to read for two
reasons:
\begin{enumerate}
\item We have to preceed the symbol ``\texttt{*}'' with a backslash in order to prevent it
      from being interpreted as a quantifier.
\item We have to ensure that the string between  ``\texttt{/*}'' and ``\texttt{*/}'' does
      not contain the substring ``\texttt{*/}''.  The regular expression
      \\[0.2cm]
      \hspace*{1.3cm}
      \verb"([^*]|\*+[^*/])*"
      \\[0.2cm]
      specifies this substring:  This substring can have an arbitrary number of parts
      that satisfy the follwing specification:
      \begin{enumerate}
      \item A part may consists of any character different from
            the character ``\texttt{*}''.  This is specificed by the regular expression
            \verb"'[^*]'".
      \item A part may be a sequence of ``\texttt{*}'' characters
            that is neither followed by the character ``\texttt{/}'' nor the character
            ``\texttt{*}''.   These parts are matched by the regular expression
            \verb"\*+[^*/]".
      \end{enumerate}
      Concatenating any number of these parts will never produce a string containing the
      substring ``\texttt{*/}''.
\end{enumerate}

\begin{figure}[!ht]
\centering
\begin{Verbatim}[ frame         = lines, 
                  framesep      = 0.3cm, 
                  firstnumber   = 1,
                  labelposition = bottomline,
                  numbers       = left,
                  numbersep     = -0.2cm,
                  xleftmargin   = 0.3cm,
                  xrightmargin  = 0.3cm,
                ]
    printComments := procedure(file) {
        lines := readFile(file);
        for (l in lines) {
            match (l) {
                regex '\s*(//.*)'                       as c: print(c[2]);
                regex '\s*(/\*([^*]|\*+[^*/])*\*+/)\s*' as c: print(c[2]);
            }
        }
    };
    
    for (file in params) {
        printComments(file);
    }
\end{Verbatim}
\vspace*{-0.3cm}
\caption{Extracting comments from a given file.}
\label{fig:find-comments.stlx}
\end{figure}

If the code shown in Figure \ref{fig:find-comments.stlx} is stored in the file
\texttt{find-comments.stlx}, then we can invoke this program as
\\[0.2cm]
\hspace*{1.3cm}
\texttt{setlx find-comments.stlx --params file.stlx}
\\[0.2cm]
The option ``\texttt{--params}'' creates the global variable \texttt{params} that contains
a list of length one.  The first element of this list is the string
``\texttt{file.stlx}''.  Therefore, the \texttt{for} loop in line 11 will call the
function \texttt{printComments} with the string ``\texttt{file.stlx}''.  If we invoke the
program using the command
\\[0.2cm]
\hspace*{1.3cm}
\texttt{setlx find-comments.stlx --params *.stlx}
\\[0.2cm]
instead, then the function \texttt{printComments} will be called for all files ending in
``\texttt{.stlx}''. 

\subsection{Conditions in \texttt{match} Statements}
A clause of a \texttt{match} statement can contain an optional Boolean condition that is
separated from the regular expression by the string ``\texttt{|}''.   Figure
\ref{fig:find-palindrom.stlx} shows a program to search for palindroms in a given file.
Line 6 shows how a condition can be attached to a \texttt{regex} clause. The regular
expression 
\\[0.2cm]
\hspace*{1.3cm}
\verb|'[a-zA-Z]+'|
\\[0.2cm]
matches any number of letters, but the string \texttt{c[1]} corresponding to the match
is only added to the set of palindroms if the predicate \texttt{isPalindrom[c[1])} yields
\texttt{true}. 

\begin{figure}[!ht]
\centering
\begin{Verbatim}[ frame         = lines, 
                  framesep      = 0.3cm, 
                  firstnumber   = 1,
                  labelposition = bottomline,
                  numbers       = left,
                  numbersep     = -0.2cm,
                  xleftmargin   = 0.8cm,
                  xrightmargin  = 0.8cm,
                ]
    findPalindrom := procedure(file) {
        all := split(join(readFile(file), "\n"), '[^a-zA-Z]+');
        palindroms := {};
        for (s in all) {
            match (s) {
                regex '[a-zA-Z]+' as c | isPalindrom(c[1]): 
                      palindroms += { c[1] };
                regex '.|\n': 
                      // skip rest
            }
        }
        return palindroms;        
    };
    
    isPalindrom := procedure(s) {
        n := #s + 1;
        return +/ [s[n - i] : i in [1 .. #s]] == s;
    };
\end{Verbatim}
\vspace*{-0.3cm}
\caption{A program to find palindroms in a file.}
\label{fig:find-palindrom.stlx}
\end{figure}


\section{The \texttt{scan} Statement}
The program to extract comments that was presented in the previous subsection is quite
unsatisfactory as it will only recognize those strings that span a single line.
Figure \ref{fig:find-comments-scan.stlx} shows a function that instead extracts all
comments form a given file.  The function \texttt{printComments} takes a string as
argument.  This string is interpreted as the name of a file.  In line 2, the function 
\texttt{readFile} reads this file.  The function produces a list of strings.  Each string
corresponds to a single line of the file without the trailing line break.  The function
\texttt{join} joins all these lines into a single string.  As the second argument of
\texttt{join} is the string ``\texttt{\symbol{92}n}'', a newline is put inbetween the
lines that are joined.  The end result is that the variable \texttt{s} contains the
content of the file as one string.  This string is the scanned using the \texttt{scan}
statement in line 3.  The \texttt{scan} statement is explained in more detail below.


\begin{figure}[!ht]
\centering
\begin{Verbatim}[ frame         = lines, 
                  framesep      = 0.3cm, 
                  firstnumber   = 1,
                  labelposition = bottomline,
                  numbers       = left,
                  numbersep     = -0.2cm,
                  xleftmargin   = 0.8cm,
                  xrightmargin  = 0.8cm,
                ]
    printComments := procedure(file) {
        s := join(readFile(file), "\n");
        scan (s) {
            regex '//[^\n]*'                as c: print(c[1]);
            regex '/\*([^*]|\*+[^*/])*\*+/' as c: print(c[1]);
            regex '.|\n'                        : // skip every thing else
        }
    };
\end{Verbatim}
\vspace*{-0.3cm}
\caption{Extracting comments using the match statement.}
\label{fig:find-comments-scan.stlx}
\end{figure}

The general form of a scan statement is as follows:
\\[0.2cm]
\hspace*{1.3cm} \texttt{scan ($s$) \{}  \\
\hspace*{1.8cm} \texttt{regex} $r_1$ \texttt{as} $l$ : $b_1$ \\
\hspace*{1.8cm} $\vdots$                                                  \\
\hspace*{1.8cm} \texttt{regex} $r_n$ \texttt{as} $l$ : $b_n$ \\
\hspace*{1.3cm} \texttt{\}}             
\\[0.2cm]
Here, $s$ is a string to be analyzed, the variables $r_1$, $\cdots$, $r_n$ are regular
expressions, while $b_1$, $\cdots$, $b_n$ are lists of statements.  The \texttt{scan}
statement works as follows:
\begin{enumerate}
\item All the regular expressions $r_1$, $\cdots$, $r_n$ are tried in parallel to match a
      prefix of the string $s$.
      \begin{enumerate}
      \item If none of these regular expression matches, the scan statement is aborted
            with an error message.
      \item If exactly one regular expression $r_i$ matches, then the corresponding
            statements $b_i$ are executed and the prefix matched by $r_i$ is removed from
            the beginning of $s$.
      \item If more than one regular expression matches, then there is a conflict which
            is resolved in two steps:
            \begin{enumerate}
            \item If the prefix matched by some regular expression $r_j$ is longer
                  than any other prefix matched by a another regular expression $r_i$,
                  then the regular expression $r_j$ wins, the list of statements $b_j$ is
                  executed and the prefix matched by $r_j$ is removed from $s$.
            \item If there are two (or more) regular expressions $r_i$ and $r_j$ that both
                  match a prefix of maximal length, then the regular expression with the
                  lowest index wins, i.e.~if $i < j$, then $b_i$ is executed.
            \end{enumerate}
      \end{enumerate}
\item This is repeated as long as the string $s$ is not empty.  Therefore, a \texttt{scan}
      statement is like a \texttt{while} loop combined with a \texttt{match} statement.
\end{enumerate}
The clauses in a \texttt{scan} statement can also have Boolean conditions attached.  This
works the same way as it works for a \texttt{match} statement.

The \texttt{scan} statement provides a functionality that is similar to the functionality
provided  by tools like \texttt{lex} \cite{lesk:1975} or
\href{http://jflex.de}{\textsl{JFlex}} \cite{klein:2009}. 
In order to support this claim, we present an example program that computes the marks of
an exam.  Assume the results of an exam are collected in a text file like the one shown in
Figure \ref{fig:result.txt}.  The first line of this file shows that this is an exam about
algorithms and the secomnd line gives the name of the group.  The fourth line tells us
that there are 6 exercises in the given exam and the remaining lines list the points that
have been achieved by individual students.  A hyphen signals that an exercise has not been
worked on.  In order to calculate marks, we just have to add up all the points.  From
this, the mark can easily be calculated.

\begin{figure}[!ht]
\centering
\begin{Verbatim}[ frame         = lines, 
                  framesep      = 0.3cm, 
                  firstnumber   = 1,
                  labelposition = bottomline,
                  numbers       = left,
                  numbersep     = -0.2cm,
                  xleftmargin   = 0.8cm,
                  xrightmargin  = 0.8cm,
                ]
    Exam:  Algorithms
    Group: TIT07AIX
    
    Exercises:                1. 2. 3. 4. 5. 6.
    Max M�ller-L�denscheidt:  8  9  8  -  7  6
    Daniel Dumpfbacke:        4  4  2  0  -  -
    Susi Sorglos:             9 12 12  9  9  6
    Jacky Jeckle:             9 12 12  -  9  6
\end{Verbatim}
\vspace*{-0.3cm}
\caption{Typical results from an exam.}
\label{fig:result.txt}
\end{figure}


\begin{figure}[!ht]
\centering
\begin{Verbatim}[ frame         = lines, 
                  framesep      = 0.3cm, 
                  firstnumber   = 1,
                  labelposition = bottomline,
                  numbers       = left,
                  numbersep     = -0.2cm,
                  xleftmargin   = 0.8cm,
                  xrightmargin  = 0.8cm,
                ]
    evalExam := procedure(file, maxPoints) {
        all := join(readFile(file), "\n");
        state := "normal";
        scan(all) using map {
            regex '[a-zA-Z]+:.*\n': // skip header
            regex '[A-Za-z�������]+\s[A-Za-z�������\-]+:' as [ name ]:
                  nPrint(name);
                  state := "printBlanks";
                  sumPoints := 0;
            regex '[ \t]+' as [ whiteSpace ] | state == "printBlanks":
                  nPrint(whiteSpace);  
                  state := "normal";
            regex '[ \t]+' | state == "normal": 
                  // skip white space between points
            regex '0|[1-9][0-9]*' as [ number ]:
                  sumPoints += int(number);
            regex '-': 
                  // skip exercises that have not been done  
            regex '\n' | sumPoints != om:
                  print(mark(sumPoints, maxPoints));
                  sumPoints := om;
            regex '[ \t]*\n' | sumPoints == om:
                  // skip empty lines
            regex '.|\n' as [ c ]:
                  print("unrecognized character: $c$");
                  print("line:   ", map["line"]);
                  print("column: ", map["column"]);
        }
    };
    mark := procedure(p, m) {
        return 7.0 - 6.0 * p / m;
    };    
\end{Verbatim}
\vspace*{-0.3cm}
\caption{A program to compute marks for an exam.}
\label{fig:exam.stlx}
\end{figure}

Figure \ref{fig:exam.stlx} shows a program that does this calculation.  We discuss this
program line by line.
\begin{enumerate}
\item The function \texttt{evalExam} takes two arguments:  The first is the name of a file
      containing the results of the exam and the second argument is the number of points 
      needed to get the best mark.  This number is a parameter that is needed to calculate
      the marks.
\item Line 2 creates a string containing the content of the given file.
\item We will use two states in our scanner:
      \begin{enumerate}
      \item The first state is identified by the string \texttt{"normal"}.
            Initially, the variable \texttt{state} has this value.
      \item The second state is identified by the string \texttt{"printBlanks"}. 
            We enter this state when we have read the name of a student.
            This state is needed to read the white space between the name of a student 
            and the first number following the student.

            In state \texttt{"normal"}, all white space is discarded, but in state
            \texttt{"printBlanks"} white space is printed.  This is necessary to format
            the output.
      \end{enumerate}
      Line 3 therefore initializes the variable \texttt{state} to \texttt{"normal"}.
\item The regular expression in line 5 is needed to skip the header of the file.  These
      header lines can be recognized that there is a single word followed by a colon
      `\texttt{:}''.   In contrast, the names of students always consist of two words.
\item The regular expression inline 5 matches the name of a student. 
      This name is printed and the number of points for this student is set to $0$.
      Furthermore, when the name of a student is seen, the state is changed to the state
      \texttt{"printBlanks"}.
\item The regular expression in line 10 matches the white space following the name of a
      student.  This white space is then printed and the state is switched back to
      \texttt{"normal"}.
\item Line 13 skips over white space that is encountered in state \texttt{"normal"}.
\item The clause in line 15 recognizes strings that can be interpreted as numbers.  These
      strings are converted into numbers with the help of the function \texttt{int} and
      then this number is added to the number of points achieved by this student.
\item Line 17 skips hyphens as these correspond to an exercise that has not been attempted
      by the student.
\item Line 19 checks whether we are at the end of a line listingthe points of a student.
      This is the case if we encounter a newline and the variable \texttt{sumPoints}
      is not undefined.  In this case, the mark for this student is computed and printed.
      Furthermore, the variable \texttt{sumPoints} is set back to the undefined status.
\item Any empty lines are skipped in line 22.
\item Finally, if we encounter any remaing character, then there is a syntax error
      in our input file.  In this case, line 24 recognizes this character and
      produces an error message.  This error message
      specifies the line and column of the character.  This is done with the help of the
      variable \texttt{map} that has been declared in line 4 via the \texttt{using}
      directive.  
\end{enumerate}

\section{Additional Functions Using Regular Expressions}
There are three predefined functions that use regular expressions.
\begin{enumerate}
\item The function 
      \\[0.2cm]
      \hspace*{1.3cm}
      $\mathtt{matches}(s, r)$ 
      \\[0.2cm]
      takes a string $s$ and a regular expression $r$ as its arguments.  It returns
      \texttt{true} if the string $s$ is in the language described by the regular
      expression $r$.  For example, the expression
      \\[0.2cm]
      \hspace*{1.3cm}
      \texttt{matches(\symbol{34}42\symbol{34}, '0|[1-9][0-9]*');}
      \\[0.2cm]
      returns \texttt{true} as the string \texttt{\symbol{34}42\symbol{34}} can be
      interpreted as a number and the regular expression \texttt{'0|[1-9][0-9]*'}
      describes natural numbers in the decimal system.

      There is a variant of \texttt{matches} that takes three arguments.  It is called as
      \\[0.2cm]
      \hspace*{1.3cm}
      $\texttt{matches}(s, r, \mathtt{true})$.
      \\[0.2cm]
      In this case, $r$ should be a regular expression containing serveral \emph{groups},
      i.e.~there should be several subexpressions in $r$ that are enclosed in
      parentheses.  Then, if $r$ matches $s$, the function \texttt{matches} returns a list
      of substrings of $s$.  The first element of this list is  the  string $s$,
      the remaining elements are the substrings corresponding to the different groups of $r$.
      For example, the expression
      \\[0.2cm]
      \hspace*{0.3cm}
      \verb|matches("+49-711-6673-4504", '\+([0-9]+)-([0-9]+)-([0-9]+)-([0-9]+)', true)|
      \\[0.2cm]
      returns the list
      \\[0.2cm]
      \hspace*{1.3cm}
      \texttt{["+49-711-6673-4504", "49", "711", "6673", "4504"]}.
      \\[0.2cm]
      If \texttt{matches} is called with three arguments where the last argument is
      \texttt{true}, an unsuccessful match returns the empty list.
\item The function
      \\[0.2cm]
      \hspace*{1.3cm}
      \texttt{replace($s$, $r$, $t$)}
      \\[0.2cm]
      receives three arguments: the arguments $s$ and $t$ are strings, while $r$ is a
      regular expression.  The function looks for substrings in $s$ that match $r$.  These
      substrings are then replaced by $t$.  For example, the expression
      \\[0.2cm]
      \hspace*{1.3cm}
      \verb|replace("+49-711-6673-4504", '[0-9]{4}', "XXXX");| 
      \\[0.2cm]
      returns the string
      \\[0.2cm]
      \hspace*{1.3cm}
      \verb|"+49-711-XXXX-XXXX"|.
\item There is a variant to the function \texttt{replace($s$, $r$, $t$)} that replaces only the first substring in $s$
      that matches $r$.  This variant is called \texttt{replaceFirst} and is called as
      \\[0.2cm]
      \hspace*{1.3cm}
      \texttt{replaceFirst($s$, $r$, $t$)}.
      \\[0.2cm]
      For example, the expression
      \\[0.2cm]
      \hspace*{1.3cm}
      \verb|replaceFirst("+49-711-6673-4504", '[0-9]{4}', "XXXX");| 
      \\[0.2cm]
      returns the string
      \\[0.2cm]
      \hspace*{1.3cm}
      \verb|"+49-711-XXXX-4504"|.
\end{enumerate}




%%% Local Variables: 
%%% mode: latex
%%% TeX-master: "tutorial"
%%% End: 
