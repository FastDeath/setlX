\chapter{Whetting Your Appetite}
This chapter contains a short overview of the data types supported by
\setlx\ and tries to whet your appetite for the language by showing off some of
the features that are unique to \setlx.  Before we discuss the more elaborate data types,
we introduce the basic data types for numbers and strings and show how to invoke the interpreter.


\section{Getting Started}
\setlx\ is an interpreted language.  To start the interpreter,  the file
\texttt{setlX} has to be both executable and part of the search path.   If these
preconditions are satisfied, the command
\\[0.2cm]
\hspace*{1.3cm}
\texttt{setlX}
\\[0.2cm]
 launches the interpreter.  The interpreter first prints the banner shown in Figure
\ref{fig:banner}, followed by a promt``\texttt{=>}''.  Commands are typed after the prompt.    
If the command is an assignment or an expression, then it is terminated by a semicolon.
However, complex commands like, for example, loops are not terminated by a semicolon.

\begin{figure}[!ht]
\centering
\begin{Verbatim}[ frame         = lines, 
                  framesep      = 0.3cm, 
                  labelposition = bottomline,
                  numbers       = none,
                  numbersep     = -0.2cm,
                  xleftmargin   = 0.0cm,
                  xrightmargin  = 0.0cm,
                ]
-====================================setlX=============================v2.0.2=-

Welcome to the setlX interpreter!

Open Source Software from http://setlX.randoom.org/
(c) 2011-2013 by Herrmann, Tom

You can display some helpful information by using '--help' as parameter when
launching this program.

Interactive-Mode:
  The 'exit;' statement terminates the interpreter.

-===============================Interactive=Mode==============================-

=> 
\end{Verbatim}
\vspace*{-0.3cm}
\caption{The \setlx\ banner followed by a prompt.}
\label{fig:banner}
\end{figure}
\noindent
The welcome banner of \textsc{SetlX} points out that we can call \texttt{setlX} with the parameter `
`\texttt{--help}''.   Doing this yields the output shown in Figure \ref{fig:help-option} below. 

\begin{figure}[!ht]
\centering
\begin{Verbatim}[ frame         = lines, 
                  framesep      = 0.3cm, 
                  firstnumber   = 1,
                  labelposition = bottomline,
                  numbers       = none,
                  numbersep     = -0.2cm,
                  xleftmargin   = 0.0cm,
                  xrightmargin  = 0.0cm,
                ]
-====================================setlX=============================v2.0.2=-

File paths supplied as parameters for this program will be parsed and executed.
The interactive mode will be started if called without any file parameters.

Interactive-Mode:
  The 'exit;' statement terminates the interpreter.

Additional parameters:
  --ev <expression>
     evaluates next argument as expression and exits
  --ex <statement>
     executes next argument as statement and exits
  --libraryPath <path>
     override SETLX_LIBRARY_PATH environment variable
  --multiLineMode
     only accept input in interactive mode after additional new line
  --noAssert
      disables all assert functions
  --noExecution
      load and check code for syntax errors, but do not execute it
  --params <argument> ...
     passes all following arguments to executed program via `params' variable
  --predictableRandom
      always use same random sequence (debugging)
  --real32
  --real64
  --real128
  --real256
      sets the width of the real-type in bits (real64 is the default)
  --realDefault
  --realEngineering
  --realPlain
      sets how the exponent of reals is displayed
  --verbose
      display the parsed program before executing it
  --version
      displays the interpreter version and terminates
\end{Verbatim}
\vspace*{-0.3cm}
\caption{Calling \texttt{setlx} with the option ``\texttt{--help}''.}
\label{fig:help-option}
\end{figure}


\noindent
The \setlx\  interpreter can be used as a simple calculator:
Typing 
\\[0.2cm]
\hspace*{1.3cm}
\texttt{=>} \texttt{\textsl{1/3 + 2/5;}}
\\[0.2cm]
and then hitting the return key\footnote{
In \setlx, statements can extend over many lines.  If the user intends to use multiline
statements, then she can start the interpreter using the commandline switch ``\texttt{--multiLineMode}''.
In multi line mode, the return key needs to be hit twice to signal
the end of the input.  Instead of using a commandline switch, the user can also issue the command
\\
\hspace*{1.3cm}
\texttt{multiLineMode(true);}
\\
to activate multiline mode.
To switch back to single line mode, use the command ``\texttt{multiLineMode(false);}''.
} 
will result in the following response:
\\[0.2cm]
\hspace*{1.3cm}
\texttt{\symbol{126}< Result: 11/15 >\symbol{126}}
\\[0.2cm]
After printing the result, the interpreter provides a new prompt so the next command can
be entered.  Incidently, the last 
example shows the first data type supported by \setlx: rational numbers.  A rational number
consists of a nominator and a denominator, both of which are integers.  \setlx\ takes care to
ensure that nominator and denominator are always in lowest terms.  Furthermore, if the
denominator has the value 1, only the nominator ist printed.  Therefore, after typing
\\[0.2cm]
\hspace*{1.3cm}
\texttt{=>} \texttt{\textsl{1/3 + 2/3;}}
\\[0.2cm]
\setlx\ will respond:
\\[0.2cm]
\hspace*{1.3cm}
\texttt{\symbol{126}< Result: 1 >\symbol{126}}
\\[0.2cm]
The precision of rational numbers is only limited by the available memory.  For example, the command
\\[0.2cm]
\hspace*{1.3cm}
\cmd{50!;}
\\[0.2cm]
computes the factorial of $50$ and yields the result
\\[0.2cm]
\hspace*{0.1cm}
\texttt{\symbol{126}< Result: 30414093201713378043612608166064768844377641568960512000000000000 >\symbol{126}}.
\\[0.2cm]
If you prefer to calculate with floating point values, the easiest way is to add
\texttt{0.0} at the end of an expression because if an arithmetic expression 
contains a floating point value, the result is converted to a floating point number.
Therefore, the command
\\[0.2cm]
\hspace*{1.3cm}
\texttt{=> \textsl{1/3 + 2/5 + 0.0;}}
\\[0.2cm]
yields the answer:
\\[0.2cm]
\hspace*{1.3cm}
\texttt{\symbol{126}< Result: 0.7333333333333333 >\symbol{126}}
\\[0.2cm]
Of course, the same can also be achieved via the command
\\[0.2cm]
\hspace*{1.3cm}
\cmd{1.0/3 + 2/5;}
\\[0.2cm]
Since the precision of rational numbers in \setlx\ is not limited, we can do things like computing
$\sqrt{2}$ to a hundred decimal places.  Figure \ref{fig:sqrt.stlx} on page \pageref{fig:sqrt.stlx}
shows a program that uses the 
\href{http://en.wikipedia.org/wiki/Methods_of_computing_square_roots#Babylonian_method}{\emph{Babylonian method}}
to compute the square root of $2$.  The idea is to compute $\sqrt{2}$ as the limit of the sequence $(b_n)_n$ which is defined as
\\[0.2cm]
\hspace*{1.3cm}
$b_0 := 2$ \quad and \quad $\displaystyle b_{n+1} := \frac{1}{2} \cdot \left(b_n + \frac{2}{b_n}\right)$.
\\[0.2cm]
In order to print a rational number in decimal notation with a fixed number of places, the function
$\texttt{nDecimalPlaces}(x,n)$ has been used.  The first argument $x$ is the rational number to be
printed, while the second argument $n$ gives the number of places to be printed.  To run this
program from the command line, assume it is stored in a file with the name \texttt{sqrt.stlx} in the
current directory.  Then the command
\\[0.2cm]
\hspace*{1.3cm}
\texttt{setlX sqrt.stlx}
\\[0.2cm]
loads and executes this program.  Alternatively, the command can be executed interactively in the
interpreter via the following command:
\\[0.2cm]
\hspace*{1.3cm}
\cmd{load(\symbol{34}sqrt.stlx\symbol{34});}
\\[0.2cm]
The output of the program is shown in Figure \ref{fig:sqrt-output}.

\begin{figure}[!ht]
\centering
\begin{Verbatim}[ frame         = lines, 
                  framesep      = 0.3cm, 
                  firstnumber   = 1,
                  labelposition = bottomline,
                  numbers       = left,
                  numbersep     = -0.2cm,
                  xleftmargin   = 0.8cm,
                  xrightmargin  = 0.8cm,
                ]
    b := 2;
    for (n in [1 .. 9]) {
        b := 1/2 * (b + 2/b);
        print(n + ": " + nDecimalPlaces(b, 100));
    }
\end{Verbatim}
\vspace*{-0.3cm}
\caption{A program to calculate $\sqrt{2}$ to 100 places.}
\label{fig:sqrt.stlx}
\end{figure}


\begin{figure}[!ht]
\centering
\begin{Verbatim}[ fontsize      = \scriptsize,
                  frame         = lines, 
                  framesep      = 0.3cm, 
                  firstnumber   = 1,
                  labelposition = bottomline,
                  numbers       = none,
                  numbersep     = -0.2cm,
                  xleftmargin   = 0.5cm,
                  xrightmargin  = 0.5cm,
                ]
1: 1.5000000000000000000000000000000000000000000000000000000000000000000000000000000000000000000000000000
2: 1.4166666666666666666666666666666666666666666666666666666666666666666666666666666666666666666666666666
3: 1.4142156862745098039215686274509803921568627450980392156862745098039215686274509803921568627450980392
4: 1.4142135623746899106262955788901349101165596221157440445849050192000543718353892683589900431576443402
5: 1.4142135623730950488016896235025302436149819257761974284982894986231958242289236217849418367358303565
6: 1.4142135623730950488016887242096980785696718753772340015610131331132652556303399785317871612507104752
7: 1.4142135623730950488016887242096980785696718753769480731766797379907324784621070388503875343276416016
8: 1.4142135623730950488016887242096980785696718753769480731766797379907324784621070388503875343276415727
9: 1.4142135623730950488016887242096980785696718753769480731766797379907324784621070388503875343276415727
\end{Verbatim}
\vspace*{-0.3cm}
\caption{The output produced by the program in Figure \ref{fig:sqrt.stlx}.}
\label{fig:sqrt-output}
\end{figure}

\noindent
The last program also demonstrates that \setlx\ supports strings.  In \setlx, any sequence of
characters enclosed in double quote characters is a string.  For example, the 
\emph{hello world progam} in \setlx\ is just 
\\[0.2cm]
\hspace*{1.3cm}
\texttt{=> \textsl{\symbol{34}Hello world!\symbol{34};}}
\\[0.2cm]
It yields the output:
\\[0.2cm]
\hspace*{1.3cm}
\texttt{\symbol{126}< Result: "Hello world!" >\symbol{126}}
\texttt{}
\\[0.2cm]
However, this only works in interactive mode.  If you want to be more verbose\footnote{
If you want to be much more verbose, you should program in either \textsl{Cobol} or \textsl{Java}.
} or if you are not working interactively, you can instead write
\\[0.2cm]
\hspace*{1.3cm}
\texttt{=> \textsl{print(\symbol{34}Hello world!\symbol{34});}}
\\[0.2cm]
This will yield two lines of  output:
\begin{verbatim}
        Hello world!
        ~< Result: om >~
\end{verbatim}
The first lines show the effect of the invocation of the function \texttt{print}, the
second line gives the return value computed by the call of the function $\texttt{print}()$.  As the
function $\texttt{print}()$ does not return a meaningful value, the
return value is the \emph{undefined value} that, following the
tradition of \textsc{Setl}, is denoted as $\Omega$.  In \setlx, this value is written as \texttt{om}.


In order to assign the value of an expression to a variable, \setlx\ provides the assignment
operator ``\texttt{:=}''.  Syntactically, this operator is the only major deviation from the syntax
of the programming language \texttt{C}.
For example, the statement
\\[0.2cm]
\hspace*{1.3cm}
\cmd{x := 1/3;}
\\[0.2cm]
binds the variable \texttt{x} to the fraction \texttt{1/3}.  
\vspace*{0.3cm}

\noindent
\textbf{Important}:  In \setlx, the names of variables
have to start with a lower case letter.  After the first letter, the name can contain
upper or lower case letters, digits, or underscores.  If the first letter is an upper case
letter, then the identifier is interpreted as a \emph{functor}.  This notion will be
explained when discussing \emph{terms}.


\section{Boolean Values}
The Boolean values \texttt{true} and \texttt{false} represent truth and falsity.  Boolean
expressions can be constructed using the comparison operators ``\texttt{==}'', 
``\texttt{!=}'', ``\texttt{<}'',  ``\texttt{>}'', ``\texttt{<=}'', and ``\texttt{>=}''.
 \setlx\
provides the following propositional operators to combine Boolean expressions.
\begin{enumerate}
\item ``\texttt{\&\&}'' denotes the logical \emph{and} (also known as \emph{conjunction}), so the expression
      \\[0.2cm]
      \hspace*{1.3cm}
      \texttt{a \&\& b}
      \\[0.2cm]
      is true if and only if both \texttt{a} and \texttt{b} are true.
\item ``\texttt{||}'' denotes the logical \emph{or}, which is also known as \emph{disjunction}.
      so the expression
      \\[0.2cm]
      \hspace*{1.3cm}
      \texttt{a || b}
      \\[0.2cm]
      is true if either \texttt{a} or \texttt{b} is true.  Note that 
      \texttt{a || b} is true if both \texttt{a} and \texttt{b} is
      true.
\item ``\texttt{!}'' denotes the logical \emph{not}, (also known as \emph{negation}).
      Therefore, the expression
      \\[0.2cm]
      \hspace*{1.3cm}
      \texttt{!a}
      \\[0.2cm]
      is true if and only if \texttt{a} is false.  
\item ``\texttt{=>}'' denotes the logical \emph{implication}.
      Therefore, the expression
      \\[0.2cm]
      \hspace*{1.3cm}
      \texttt{a => b}
      \\[0.2cm]
      is true if either \texttt{a} is false or \texttt{b} is true. Hence the expression 
      \\[0.2cm]
      \hspace*{1.3cm}
      \texttt{a => b}
      \\[0.2cm]
      has the same truth value as the expression
      \\[0.2cm]
      \hspace*{1.3cm}
      \texttt{!a || b}.
\item ``\texttt{<==>}'' denotes the logical \emph{equivalence},
      so the expression
      \\[0.2cm]
      \hspace*{1.3cm}
      \texttt{a <==> b}
      \\[0.2cm]
      is true if either  \texttt{a} and \texttt{b} are both true or \texttt{a} and
      \texttt{b} are both false. Therefore,  the expression 
      \\[0.2cm]
      \hspace*{1.3cm}
      \texttt{a <==> b}
      \\[0.2cm]
      has the same value as the expression
      \\[0.2cm]
      \hspace*{1.3cm}
      \texttt{(a \&\& b) || (!a \&\& !b)}.
\item ``\texttt{<!=>}'' denotes the logical \emph{antivalence},
      so the expression
      \\[0.2cm]
      \hspace*{1.3cm}
      \texttt{a <!=> b}
      \\[0.2cm]
      is true if the truth values of \texttt{a} and \texttt{b} are different.
      Therefore,  the expression 
      \\[0.2cm]
      \hspace*{1.3cm}
      \texttt{a <!=> b}
      \\[0.2cm]
      is equivalent to the expression
      \\[0.2cm]
      \hspace*{1.3cm}
      \texttt{!(a <==> b)}.
\end{enumerate}
The operators ``\texttt{==}'' and ``\texttt{!=}'' can be used instead of the operators
``\texttt{<==>}'' and ``\texttt{<!=>}''. 

Furthermore, \setlx\ supports both the universal quantifier ``\texttt{forall}'' and the
existential quantifier ``\texttt{exists}''.  For example, to test whether the formula
\\[0.2cm]
\hspace*{1.3cm}
$\forall x \in \{ 1, \cdots, 10 \}: x^2 \leq 2^x$
\\[0.2cm]
is true, we can evaluate the following expression:
\\[0.2cm]
\hspace*{1.3cm}
\texttt{forall (x in \{1..10\} | x ** 2 <= 2 ** x);}
\\[0.2cm]
This expression checks whether $x^2$ less than or equal to $2^x$ for all $x$ between 1 and
10.  Syntactically, a \texttt{forall} expression is described by the following grammar
rule: 
\\[0.2cm]
\hspace*{1.3cm}
$\textsl{expr} \;\rightarrow\; \quoted{forall} \quoted{(} \textsl{var} \quoted{in}
 \textsl{setExpr} \quoted{|} \textsl{cond} \quoted{)}
$
\\[0.2cm]
Here, \textsl{var} denotes a variable and \textsl{expr} is an expression that evaluates
to a set $s$ (or a list  or a string), while \textsl{cond} is a Boolean expression.  The
\texttt{forall} expression evaluates to \texttt{true} if for all elements of $s$ the
condition \textsl{cond} evaluates to true.  There is a generalization of the grammar rule 
that allows to check several variables simultaneously, so we can write an expression like
the following:
\\[0.2cm]
\hspace*{1.3cm}
\texttt{forall (x in \{1..10\}, y in [20..30] | x < y)}.
\\[0.2cm]
If instead we want to know whether the formula
\\[0.2cm]
\hspace*{1.3cm}
$\exists x \in \{ 1, \cdots, 10 \}: 2^x < x^2$
\\[0.2cm]
is true, we have to write:
\\[0.2cm]
\hspace*{1.3cm}
\texttt{exists (x in \{1..10\} | 2 ** x < x ** 2);}
\\[0.2cm]
This expression checks whether there exists a natural number $x \in \{1, \cdots, 10\}$
such that $2^x$ is less than $x^2$.
\vspace*{0.3cm}

\noindent
\textbf{Important}:  \texttt{forall} and \texttt{exists} statements do \underline{not}
create their own local scope.  For example, the sequence of statements
\begin{verbatim}
    exists ([x, y] in {[a,b] : a in {1..10}, b in {1..10}} | 3*x - 4*y == 5);
    print("x = $x$, y = $y$");
\end{verbatim}
will print
\\[0.2cm]
\hspace*{1.3cm}
\texttt{x = 3, y = 1}.
\\[0.2cm]
This example shows that the values of the variables \texttt{x} and \texttt{y} is available
outside of the existentially quantified expression.  The execution of
a \texttt{forall} or an \texttt{exists} statement stops as soon as the
truth value of the statement is known.
This feature is quite useful if the actual value of the variable in an \texttt{exists}
statement is needed.  It is also useful for a \texttt{forall} statement in case that the
\texttt{forall} statement fails.  For example, the expression
\\[0.2cm]
\hspace*{1.3cm}
\texttt{forall (n in [1..10] | n**2 <= 2**n);}
\\[0.2cm]
evaluates to \texttt{false} and, furthermore, it sets the value of the variable \texttt{n}
to $3$, as this is the first integer in the list \texttt{[1..10]}
where the expression $n^2 \leq 2^n$ is false.

\section{Sets}
The most interesting data type provided by \setlx\ is the \emph{set} type.  A \emph{set} is
a collection that contains all of its elements exactly once.
To create a
simple set containing the numbers $1$, $2$, and $3$, we can write
\\[0.2cm]
\hspace*{1.3cm}
\cmd{\{1, 2, 3\};}
\\[0.2cm]
Note that the comparison
\\[0.2cm]
\hspace*{1.3cm}
\cmd{\{1,2,3\} == \{2,3,1\}}
\\[0.2cm]
yields the result
\\[0.2cm]
\hspace*{1.3cm}
\texttt{Result: true}
\\[0.2cm]
since the order of elements in a set does not matter.  This behaviour can lead to
results that are sometimes surprising for the novice.  For example, the command
\\[0.2cm]
\hspace*{1.3cm}
\cmd{print(\{3,2,1\});}
\\[0.2cm]
yields the following output:
\\[0.2cm]
\hspace*{1.3cm}
\texttt{\{1, 2, 3\}}
\\[0.2cm]
The reason is, that \setlx\ does not remember the ordering of elements in a set, it just
remembers the elements and sorts them internally to maximize the efficiency of looking up
elements from a set.  

To check whether a set contains a given entity as an element, \setlx\ provides the binary infix
operator ``\texttt{in}''.  For example, the command
\\[0.2cm]
\hspace*{1.3cm}
\cmd{2 in \{1,2,3\};}
\\[0.2cm]
yields the result \texttt{true}, as $2$ is indeed an element of the set $\{1,2,3\}$, while 
\\[0.2cm]
\hspace*{1.3cm}
\cmd{4 in \{1,2,3\};}
\\[0.2cm]
 returns \texttt{false}.

The data type of a set would be quite inconvenient to use if we could only create sets by
explicitly listing all their elements.  Fortunately, there are some more powerful expressions to
create sets.  The most straightforward of them is the \emph{range operator} that can create a set
containing all the integers in a given range.  For example, to create the set containing
the integers from 1 up to 16, we can write 
\\[0.2cm]
\hspace*{1.3cm}
\cmd{\{1..16\};}
\\[0.2cm]
This will give the result
\\[0.2cm]
\hspace*{1.3cm}
\texttt{Result: \{1, 2, 3, 4, 5, 6, 7, 8, 9, 10, 11, 12, 13, 14, 15, 16\}}.
\\[0.2cm]
Here, we have used the range operator ``\texttt{..}''.  For given integers $a$ and $b$
such that $a \leq b$,
the expression
\\[0.2cm]
\hspace*{1.3cm}
\texttt{\{$a$..$b$\}}
\\[0.2cm]
will generate the set of all integers starting from $a$ up to and including the number
$b$.  Mathematically, the semantics of this operator is  given by the formula
\\[0.2cm]
\hspace*{1.3cm}
\texttt{\{$a$..$b$\} = $\{ a + n \mid a \in \mathbb{N} \wedge a + n \leq b\}$.}
\\[0.2cm]
If $b$ is less than $a$, the expression 
\\[0.2cm]
\hspace*{1.3cm}
\texttt{\{$a$..$b$\}}
\\[0.2cm]
denotes the empty set $\{\}$.

While successive elements of a set created by the expression ``\texttt{\{$a$..$b$\}}''
differ by $1$, there is a variant of the range operator that allows us to
specify the size of the difference between successive elements.   For example, the expression
\\[0.2cm]
\hspace*{1.3cm}
\cmd{\{1,3..10\};}
\\[0.2cm]
yields
\\[0.2cm]
\hspace*{1.3cm}
\texttt{Result: \{1, 3, 5, 7, 9\}}.
\\[0.2cm]
In general, when a set definition of the form
\\[0.2cm]
\hspace*{1.3cm}
\texttt{\{$a$,$b$..$c$\}}
\\[0.2cm]
is evaluated, there are two cases.
\begin{enumerate}
\item $a < b$.  In this case, the \textsl{step size} $s$ is defined as
      \\[0.2cm]
      \hspace*{1.3cm}
      $s := b - a$.
      \\[0.2cm]
      Then, we have
      \\[0.2cm]
      \hspace*{1.3cm}
      $\texttt{\{$a$,$b$..$c$\}} := 
       \{ a + n \cdot s \mid n \in \mathbb{N} \wedge a + n \cdot s \leq c  \}
      $.
\item $a > b$.  In this case, the \textsl{step size} $s$ is defined as
      \\[0.2cm]
      \hspace*{1.3cm}
      $s := a - b$.
      \\[0.2cm]
      Then, we have
      \\[0.2cm]
      \hspace*{1.3cm}
      $\texttt{\{$a$,$b$..$c$\}} := 
       \{ a + n \cdot s \mid n \in \mathbb{N} \wedge a + n \cdot s \geq c  \}
      $.
      \\[0.2cm]
      For example, we have
      \\[0.2cm]
      \hspace*{1.3cm}
      \texttt{\{10, 8..1\} = \{10, 8, 6, 4, 2\}}.
\end{enumerate}

\subsection{Operators on Sets}
The most basic operators on sets are as follows:
\begin{enumerate}
\item ``\texttt{+}'' is used to compute the \emph{union} of two sets.

      In mathematics, the union of two sets $s_1$ and $s_2$ is  written
      as $s_1 \cup s_2$.  It is defined as
      \\[0.2cm]
      \hspace*{1.3cm}
      $s_1 \cup s_2 := \{ x \mid x \in s_1 \vee x \in  s_2 \}$.
\item ``\texttt{*}'' computes the \emph{intersection} of two sets. 

      In mathematics, the intersection of two sets $s_1$ and $s_2$ is  written
      as $s_1 \cap s_2$.  It is defined as
      \\[0.2cm]
      \hspace*{1.3cm}
      $s_1 \cap s_2 := \{ x \mid x \in s_1 \wedge x \in  s_2 \}$.
\item ``\texttt{-}'' computes the \emph{difference} of two sets.

      In mathematics, the difference of two sets $s_1$ and $s_2$ is often written
      as $s_1 \symbol{92} s_2$.  It is defined as
      \\[0.2cm]
      \hspace*{1.3cm}
      $s_1 \symbol{92} s_2 := \{ x \in s_1 \mid x \notin \}$.
\item ``\texttt{><}'' computes the \emph{Cartesian product} of two sets.

      In mathematics, the Cartesian product of two sets $s_1$ and $s_2$ is written
      as $s_1 \times s_2$.  It is defined as
      \\[0.2cm]
      \hspace*{1.3cm}
      $s_1 \times s_2 := \{ \pair(x_1,x_2) \mid x_1 \in s_1 \wedge x_2 \in s_2 \}$.
\item ``\texttt{** 2}'' computes the \emph{Cartesian product} of a set with itself,
      i.e.~we have 
      \\[0.2cm]
      \hspace*{1.3cm}
      $s \;\mathtt{**}\; 2 := s \times s$.
\item ``\texttt{2 ** }'' computes the \emph{power set} of a given set.  For a given set
      $s$, the power set of $s$ is defined as the set of all subsets of $s$.
      In mathematics, the power set of a given set $s$ is written as $s^2$ and the formal
      definition is
      \\[0.2cm]
      \hspace*{1.3cm}
      $2^s := \{ m \mid m \subseteq s \}$.
      \\[0.2cm]
      For example, the expression
      \\[0.2cm]
      \hspace*{1.3cm}
      \texttt{2 ** \{1,2,3\}}
      \\[0.2cm]
      yields the result
      \\[0.2cm]
      \hspace*{1.3cm}
      \texttt{\{\{\}, \{1\}, \{1, 2\}, \{1, 2, 3\}, \{1, 3\}, \{2\}, \{2, 3\}, \{3\}\}}.
\item ``\texttt{\symbol{37}}'' computes the \emph{symmetric difference} of sets.

      In mathematics, the symmetric difference of two sets $s_1$ and $s_2$ is often written
      as $s_1 \triangle s_2$.  It is defined as
      \\[0.2cm]
      \hspace*{1.3cm}
      $s_1 \triangle s_2 := (s_1 \backslash s_2) \cup (s_2 \backslash s_1)$.
\end{enumerate}
Therefore, the commands
\\[0.2cm]
\hspace*{1.3cm}
\texttt{s1 := \{ 1, 2 \}; s2 := \{ 2, 3 \};}
\\
\hspace*{1.3cm}
\texttt{print("s1 + \ s2 = \symbol{36}s1 + \ s2\symbol{36}");}
\\
\hspace*{1.3cm}
\texttt{print("s1 - \ s2 = \symbol{36}s1 - \ s2\symbol{36}");}
\\
\hspace*{1.3cm}
\texttt{print("s1 * \ s2 = \symbol{36}s1 * \ s2\symbol{36}");}
\\
\hspace*{1.3cm}
\texttt{print("s1 ** \ 2  = \symbol{36}s1 ** 2\ \symbol{36}");}
\\
\hspace*{1.3cm}
\texttt{print("2 \ ** s2  = \symbol{36}2 \ ** s2\symbol{36}");}
\\
\hspace*{1.3cm}
\texttt{print("s1 >< s2 = \symbol{36}s1 >< s2\symbol{36}");}
\\
\hspace*{1.3cm}
\texttt{print("s1 \symbol{37} \ s2 = \symbol{36}s1 \symbol{37} \ s2\symbol{36}");}
\\[0.2cm]
will produce the follwing results:
\begin{verbatim}
        s1 +  s2 = {1, 2, 3}
        s1 -  s2 = {1}
        s1 *  s2 = {2}
        s1 **  2 = {[1, 1], [1, 2], [2, 1], [2, 2]}
        2  ** s2 = {{}, {1}, {1, 3}, {3}}
        s1 >< s2 = {[1, 2], [1, 3], [2, 2], [2, 3]}
        s1 %  s2 = {1, 3}
\end{verbatim}
For sets, the exponentiation operator ``\texttt{**}'' is only defined if either the base
of the exponentiation is a set and the exponent is the number $2$ or the base is the
number $2$ and the exponent is a set.  Therefore, if $s$ is a set, expressions like
\\[0.2cm]
\hspace*{1.3cm}
\texttt{3 ** $s$} \quad and \quad \texttt{$s$ ** 3}
\\[0.2cm]
are undefined. 
Also note the use of \emph{string interpolation} in the examples given
above:  Inside a string, any expression enclosed in
\symbol{36}-symbols is evaluated and the result of this evaluation is
inserted into the string.
Strings interpolation will be discussed in more detail in the section on strings.

For all of the binary operators discussed above there is a variant of
the assignment operator that incorporates the binary operator.  For example, we can write
\\[0.2cm]
\hspace*{1.3cm}
\texttt{s += \{x\};}
\\[0.2cm]
to add the element \texttt{x} to the set \texttt{s}, while the command
\\[0.2cm]
\hspace*{1.3cm}
\texttt{s -= \{x\};}
\\[0.2cm]
removes the element \texttt{x} from the set \texttt{s}.  Of course,
these modified assignment operators also work for numbers.
\vspace*{0.2cm}

In addition to the basic operators,
\setlx\ provides a number of more elaborate operators for sets.  One of these operators is
the cardinality operator ``\texttt{\#}'', which computes the number of elements of a 
given set.  The cardinality operator is used as a unary prefix operator, for example we can write:
\\[0.2cm]
\hspace*{1.3cm}
\texttt{\# \{5,7,13\}};
\\[0.2cm]
Of course, in this example the result will be $3$.
We can sum the elements of a set using the prefix operator
``\texttt{+/}''.  For example, in order to
compute the sum 
\\[0.2cm]
\hspace*{1.3cm}
$\sum\limits_{i=1}^{6^2} i$,
\\[0.2cm]
we can create the set \texttt{\{1..6**2\}} containing all the numbers
to sum and then use the command
\\[0.2cm]
\hspace*{1.3cm}
\texttt{+/ \{1..6**2\}; }
\\[0.2cm]
to calculate the actual sum.
For arguments that are numbers, ``\texttt{**}'' is the exponentiation operator.
Therefore, the expression given above 
first computes the set of all numbers from 1 up to the number 36 and then returns the sum
of all these numbers.  There is also a binary version of the operator ``\texttt{+/}'' that
is used as an infix operator.  For a number $x$ and a set $s$, the expression
\\[0.2cm]
\hspace*{1.3cm}
\texttt{$x$ +/ s}
\\[0.2cm]
will insert $x$ into the set $s$ if the set $s$ is empty.  After that, it returns the sum of all elements in the
resulting set.  This is usefull as the expression ``\texttt{+/ \{\}}'' is undefined.  So
if we want to compute the sum of all numbers in $s$ but we are not sure whether $s$ might
be empty, we can use the expression
\\[0.2cm]
\hspace*{1.3cm}
\texttt{0 +/ s}
\\[0.2cm]
since inserting $0$ into the empty set guarantees that the result is 0 in case  the set
\texttt{s} is empty.

There is a similar operator for multiplying the elements of a set: It is the operator
``\texttt{*/}''.
For example, in order to compute the factorial\footnote{
  The factorial operator ``\texttt{!}'' is a builtin postfix operator.  Computing
  the factorial of a number using this operator is more efficient than first building a set
  and then using the operator ``\texttt{*/}''.
}
$n! = 1 \cdot 2 \cdot 3 \cdot {\dots} \cdot (n-1) \cdot n$ we can use the expression
\\[0.2cm]
\hspace*{1.3cm}
\texttt{*/ \{1..n\}}.
\\[0.2cm]
Again, there is also a binary version of this operator.  Since the number $1$ is the
neutral element for  multiplication, the first argument to this operator will most often
we one.   Therefore, the expression
\\[0.2cm]
\hspace*{1.3cm}
\texttt{1 */ s}
\\[0.2cm]
 multiplies all the numbers from a the set $s$ and this expression will return $1$ if the
 set $s$ happens to be empty.

\subsection{Set Comprehensions}
We can use \emph{set comprehension} to build sets.  For example, the command
\\[0.2cm]
\hspace*{1.3cm}
\texttt{\{ a * b : a in \{ 1 .. 3 \}, b in \{ 1 .. 3 \} \};}
\\[0.2cm]
computes the set of all products $a * b$ where both $a$ and $b$ run from $1$ to $3$.
The command will therefore compute the set
\\[0.2cm]
\hspace*{1.3cm}
\texttt{\{1, 2, 3, 4, 6, 9\}}.
\\[0.2cm]
In general, a \emph{set comprehension expression} has the form
\\[0.2cm]
\hspace*{1.3cm}
\texttt{\{ $\textsl{expr}$ :$\;x_1$ in $s_1$, $\cdots$, $x_n$ in $s_n$ | $\textsl{cond}$ \}}.
\\[0.2cm]
Here, \textsl{expr} is some expression containing the variables $x_1$, $\cdots$, $x_n$, while
$s_1$, $\cdots$, $s_n$ are either sets or lists (to be discussed later), and $\textsl{cond}$ is an
expression returning a Boolean value.  
The  set comprehension expression given above evaluates $\textsl{expr}$ for all possible combinations of 
values $x_1 \in s_1$, $\cdots$, $x_n \in s_n$ such that $\textsl{cond}$ is true and it will add the
corresponding value of $\textsl{expr}$ into the resulting set.
The expression $\textsl{cond}$ is optional.  If it is missing, it is implicitly taken to be
always true.

We are now ready to demonstrate some of the power that comes with sets.  
The following two statements compute the set of all prime numbers smaller than $100$:
\\[0.2cm]
\hspace*{1.3cm}
\texttt{s := \{2..100\};}
\\
\hspace*{1.3cm}
\texttt{s - \{ p * q : p in s, q in s \};}
\\[0.2cm]
The expression given above computes the set of prime numbers smaller
than $100$ correctly, as a prime number
is any number bigger than 1 that is not a proper product.  Therefore, if we subtract the
set of all proper products from the set of numbers, we get the set of prime numbers.
Obviously, this is not an effcient way to calculate primes, but efficiency is not the
point of this example.  It rather shows that \setlx\ is well suited to execute a
mathematical definition as it is.

Let us look at another example of set comprehension:  The expression
\\[0.2cm]
\hspace*{1.3cm}
\texttt{\{ p:p in \{2..100\} | \{ t:t in \{2..p-1\} | p \% t == 0 \} == \{\} \}}
\\[0.2cm]
yields the set of all prime numbers less than 100.  The subexpression
\\[0.2cm]
\hspace*{1.3cm}
\texttt{\{ t:t in \{2..p-1\} | p \% t == 0 \}}
\\[0.2cm]
computes the set of all those natural numbers bigger than 1 and less than \texttt{p} that divide
\texttt{p} evenly.  If this set is empty, then, by the definition of a prime number, \texttt{p} is prime.

The syntax for building set comprehensions is a little bit more
general than discussed above.  Provided $s_1$, $\cdots$, $s_n$ are sets of \emph{lists},
where all of the lists in the set $s_i$ have the same length $m_i$, a set comprehension expression
can take the form
\\[0.2cm]
\hspace*{1.3cm}
\texttt{\{ $\textsl{expr}$ :$\;[x_1,x_2,\cdots,x_{m_1}]$ in $s_1$, $\cdots$, $[z_1,z_2,\cdots, z_{m_n}]$ in $s_n$ | $\textsl{cond}$ \}}.
\\[0.2cm]
This kind of set comprehension expressions will be discussed further when discussing lists.

\subsection{Miscellaneous Set Functions}
In addition to the operators provided for sets, \setlx\ has a number of functions targeted at sets.
The first of these functions is \texttt{arb}.  For a set $s$,  an expression of the form
\\[0.2cm]
\hspace*{1.3cm}
\texttt{arb($s$)}
\\[0.2cm]
yields some element of the set $s$.  The element returned is not specified, so the expression
\\[0.2cm]
\hspace*{1.3cm}
\texttt{arb(\{1,2,3\});}
\\[0.2cm]
might yield either $1$, $2$, or $3$. 
 The function \texttt{from} has a similar effect, so
\\[0.2cm]
\hspace*{1.3cm}
\texttt{from($s$)}
\\[0.2cm]
also returns some element of the set $s$.  In addition, this element is removed from the set.
Therefore, if the set \texttt{s} is defined via
\\[0.2cm]
\hspace*{1.3cm}
\texttt{s := \{1,2,3\};}
\\[0.2cm]
and we execute the statement 
\\[0.2cm]
\hspace*{1.3cm}
\texttt{x := from(s);}
\\[0.2cm]
then some element of \texttt{s} is assigned to \texttt{x} and this element is removed form the set
the set \texttt{s}.  For a set $s$,  the expression 
\\[0.2cm]
\hspace*{1.3cm}
\texttt{first($s$)}
\\[0.2cm]
returns the \emph{first} element of the set $s$, while
\\[0.2cm]
\hspace*{1.3cm}
\texttt{last($s$)} 
\\[0.2cm]
yields the \emph{last} element.  Internally, sets are represented as ordered binary trees.  As long
as the sets contain only numbers, the ordering of these numbers is the usual ordering on
numbers. Therefore, in this case the \emph{element} is just the smalllest number and the 
\emph{last element} is the biggest number.  If the sets contain sets (or lists), these sets (or
lists) are themselves compared lexicographically.  
Things get a little bit more complicated if a set is heterogenous and contains both number and sets.
Programs should not rely on the ordering implemented in these cases.

For a given set $s$, the expression
\\[0.2cm]
\hspace*{1.3cm}
\texttt{rnd($s$)}
\\[0.2cm]
yields a random element of the set $s$.  
In contrast to the expression
``\texttt{arb($s$)}'',  this expression will in general return different results on
different invocations.  The function \texttt{rnd} also works for lists, so for a list $l$
the expression $\texttt{rnd}(l)$ returns a random element of $l$.
The function $\mathtt{rnd}()$ is also supported for integer numbers, for example the expression
\\[0.2cm]
\hspace*{1.3cm}
\texttt{rnd(5)}
\\[0.2cm]
computes a random natural number less or equal than $5$. Essentially it is equal to the expression
\texttt{rnd(\{1..5\})}.  While we are at it: The function \texttt{random} computes pseudo-random
values in the intervall $[0,1]$.  This function does not take an argument. Therefore, the expression
\\[0.2cm]
\hspace*{1.3cm}
\texttt{random()}
\\[0.2cm]
might output a value like \texttt{0.5171688974345957}. 


\section{Lists}
Besides sets, \setlx\ also supports lists.  Syntactically, the main difference is that the
curly braces ``\texttt{\{}'' and ``\texttt{\}}'' are substituted with the square brackets
``\texttt{[}'' and ``\texttt{]}''.  Semantically, a list is an ordered collection of elements that 
can contain the same element multiple times.  For example
\\[0.2cm]
\hspace*{1.3cm}
\texttt{[1,4,7,2,4,7]}
\\[0.2cm]
is a typical list.  The easiest way to construct a list is by defining the list explicitely. For
example, the assignment
\\[0.2cm]
\hspace*{1.3cm}
\texttt{l := [1,4,7,2,4,7]}
\\[0.2cm]
defines the list \texttt{[1,4,7,2,4,7]} and assigns it to the variable \texttt{l}.
Another way to construct lists is via the range operator ``\texttt{..}'':
For example, the command
\\[0.2cm]
\hspace*{1.3cm}
\texttt{[1..10]}
\\[0.2cm]
computes the list
\\[0.2cm]
\hspace*{1.3cm}
\texttt{[1, 2, 3, 4, 5, 6, 7, 8, 9, 10]}.
\\[0.2cm]
The prefix operators ``\texttt{+/}'' and ``\texttt{*/}'' work for lists in the same way that they
work for sets: ``\texttt{+/ l}'' computes the sum of all elements of the list \texttt{l}, while
``\texttt{*/ l}'' computes the product of the elements of \texttt{l}.  For example, the expression
\\[0.2cm]
\hspace*{1.3cm}
\texttt{+/ [1..100]}
\\[0.2cm]
yields the result $5050$.

\emph{List comprehension} works similar to set comprehension, the expression
\\[0.2cm]
\hspace*{1.3cm}
\texttt{[ $\textsl{expr}$ :$\;x_1$ in $s_1$, $\cdots$, $x_n$ in $s_n$ | $\textsl{cond}$ ]}
\\[0.2cm]
 picks up all tuples 
\\[0.2cm]
\hspace*{1.3cm}
$\langle x_1, \cdots, x_n \rangle \in s_1 \times \cdots \times s_n$
\\[0.2cm]
such that \texttt{cond} is true and evaluates \texttt{expr} for the corresponding values of the
variables $x_1$, $\cdots$, $x_n$.  The value of \texttt{expr} is
then inserted into the resulting list.  Here, the $s_i$ denote either sets or lists.
For example, the following expression  computes all primes up
to 100: 
\\[0.2cm]
\hspace*{1.3cm}
\texttt{[ p : p in [2..100] | \{ x : x in \{1..p\} | p \% x == 0 \} == \{1, p\} ]}.
\\[0.2cm]
Here, the idea is that a number $p$ is prime if the set of its divisors, which is the set
of all numbers $x$ such that $p \,\texttt{\%}\, x = 0$, only contains the number $1$ and
the number $p$. 

The syntax for building lists via comprehensions is a little bit more
general than discussed above.  Provided $s_1$, $\cdots$, $s_n$ are sets or lists,
such that all of the elements of $s_i$ are lists that have the same length $m_i$, a list 
can be defined as
\\[0.2cm]
\hspace*{1.3cm}
\texttt{[ $\textsl{expr}$ :$\;[x_1,x_2,\cdots,x_{m_1}]$ in $s_1$, $\cdots$, $[z_1,z_2,\cdots, z_{m_n}]$ in $s_n$ | $\textsl{cond}$ ]}.
\\[0.2cm]
For example, assume that \texttt{r1} and \texttt{r2} are sets containing \emph{pairs} of elements, where a
\emph{pair} is just a list of length two.  In this case, \texttt{r1} and \texttt{r2} could be viewed as binary
relations.  Then the 
\href{http://en.wikipedia.org/wiki/Binary_relation}{\emph{composition}} $\mathtt{r1} \circ \mathtt{r2}$ of \texttt{r1} and
\texttt{r2} can be computed as
\\[0.2cm]
\hspace*{1.3cm}
\texttt{\{ [x,z] : [x,y] in r1, [y,z] in r2 \}}.
\\[0.2cm] 
There are two infix operators for lists: The first is the operator ``\texttt{+}'' which
concatenates its arguments.  For example
\\[0.2cm]
\hspace*{1.3cm}
\texttt{[1 .. 3] + [5 .. 10];}
\\[0.2cm]
yields 
\\[0.2cm]
\hspace*{1.3cm}
\texttt{[1, 2, 3, 5, 6, 7, 8, 9, 10]}.
\\[0.2cm]
The second infix operator is ``\texttt{*}''.  While the first argument of this operator is
a list, the second  needs to be an integer.  An expression of the form
\\[0.2cm]
\hspace*{1.3cm}
\texttt{l * n}
\\[0.2cm]
concatenates \texttt{n} copies of the list \texttt{l}.  Therefore, the expression
\\[0.2cm]
\hspace*{1.3cm}
\texttt{[1, 2, 3] * 3}
\\[0.2cm]
yields the result
\\[0.2cm]
\hspace*{1.3cm}
\texttt{[1, 2, 3, 1, 2, 3, 1, 2, 3]}.
\\[0.2cm]
The cardinality operator ``\texttt{\#}'' works for lists the same way it works for sets:
It returns the number of elements.  For example 
\\[0.2cm]
\hspace*{1.3cm}
\texttt{\# [ 7, 4, 5 ];}
\\[0.2cm]
yields the result $3$.

As the elements of a list are ordered, it is possible to extract an element from a list with respect
to its position.  In general, for a list $l$ and a positive natural number $n$, the expression
\\[0.2cm]
\hspace*{1.3cm}
\texttt{$l$[$n$]}
\\[0.2cm]
selects the $n$-th element of $l$.  Here, counting starts with $1$, so $l[1]$ is the first element
of the list $l$.  For example, after the assignment
\\[0.2cm]
\hspace*{1.3cm}
\texttt{l := [99, 88, 44];}
\\[0.2cm]
the expression 
\\[0.2cm]
\hspace*{1.3cm}
\texttt{l[2]}
\\[0.2cm]
yields the result $88$.  The right hand side of an extraction expression can also be a \emph{slicing operator}:
The expression
\\[0.2cm]
\hspace*{1.3cm}
\texttt{$l$[$a$..$b$]}
\\[0.2cm]
extracts the sublist of $l$ that starts at index $a$ and ends at index $b$.  For example,
after defining $l$ via the assignment
\\[0.2cm]
\hspace*{1.3cm}
\texttt{l := [1..100];}
\\[0.2cm]
the expression
\\[0.2cm]
\hspace*{1.3cm}
\texttt{l[5..10]}
\\[0.2cm]
yields the sublist \texttt{[5, 6, 7, 8, 9, 10]}.  There are two variants of the slicing operator.
The expression
\\[0.2cm]
\hspace*{1.3cm}
\texttt{$l$[$a$..]}
\\[0.2cm]
returns the sublist of $l$ that starts with the element at index $a$, and the expression
\\[0.2cm]
\hspace*{1.3cm}
\texttt{$l$[..$b$]}
\\[0.2cm]
computes the sublist of $l$ that starts at the first element and includes all the elements up to the
element at index $b$.

Lists of variables can be used on the left hand side of an assigment.  This feature enables the
simultaneous assignment of variables.  The statement
\\[0.2cm]
\hspace*{1.3cm}
\texttt{[x,y] := [1,2];}
\\[0.2cm]
sets the variable $x$ to 1 and $y$ to 2.  A more interesting example is the following assigment:
\\[0.2cm]
\hspace*{1.3cm}
\texttt{[y,x] := [x,y];}
\\[0.2cm]
This assigment swaps the values of $x$ and $y$.

\section{Pairs, Relations,  and Functions}
In \textsc{Setl}, a pair of the form $\langle x, y \rangle$ is represented as the list $[x,y]$.
A set of pairs can be regarded as a binary relation and the notion of a relation is a
generalization of the notion of a function.   If $r$ is a binary relation, we define the
\emph{domain} and \emph{range} of $r$ as the set containing the first and second component
of the pairs, that is we have
\\[0.2cm]
\hspace*{1.3cm}
\texttt{domain(r) = \{ x :[x,y] in r \}} \quad and \quad
\texttt{range(r)  = \{ y :[x,y] in r \}}.
\\[0.2cm]
Furthermore, if $r$ is a binary relation
such that for all pairs such that
\\[0.2cm]
\hspace*{1.3cm}
$[x,y_1] \in r$ \quad and \quad $[x,y_2] \in r$ 
\\[0.2cm]
holds, we have that $y_1 = y_2$, then $r$ is called a \emph{map} and represents a function.
Therefore, a map is a set of \texttt{[\textsl{key}, \textsl{value}]} pairs such that the
keys are unique. 
If a binary relation $r$ is a map, \setlx\ permits us to use the relation as a function:  If $r$ is a map
and $x \in \textsl{domain}(r)$, then $r[x]$ denotes the unique element $y$ such that 
$\pair(x,y) \in r$:
\\[0.2cm]
\hspace*{1.3cm}
$r[x] := \left\{
\begin{array}{ll}
  y & \mbox{if the set $\{ y \mid [x,y] \in r\}$ contains exactly one element $y$;} \\[0.2cm]
  \Omega & \mbox{otherwise}.
\end{array} \right.
$
\\[0.2cm]
The program shown below in Figure
\ref{fig:map.stlx} provides a trivial example demonstrating how maps can be used
as functions in  \setlx.  
 

\begin{figure}[!ht]
  \centering
\begin{Verbatim}[ frame         = lines, 
                  framesep      = 0.3cm, 
                  labelposition = bottomline,
                  numbers       = left,
                  numbersep     = -0.2cm,
                  xleftmargin   = 0.8cm,
                  xrightmargin  = 0.8cm,
                ]
    r := { [n, n*n] : n in {1..10} };
    print( "r[3]      = $r[3]$"      );
    print( "domain(r) = $domain(r)$" );
    print( "range(r)  = $range(r)$"  );
\end{Verbatim} 
\vspace*{-0.3cm}
\caption{Binary relations as functions.}  \label{fig:map.stlx}
\end{figure} %$

\noindent
The program computes the map \texttt{r} that represents the function 
 $x \mapsto x*x$ on the set
\\[0.2cm]
\hspace*{1.3cm}
 $\{n\in \mathbb{N} \mid 1 \leq n \wedge n \leq 10 \}$.  
\\[0.2cm] 
In line 3, the relation \texttt{r} is evaluated at $x=3$.  This is done using square brackets.
Finally, 
$\textsl{domain}(\mathtt{r})$ and $\textsl{range}(\texttt{r})$ are computed.
We get the following result:
\begin{verbatim}
        r[3]      = 9
        domain(r) = {1, 2, 3, 4, 5, 6, 7, 8, 9, 10}
        range(r)  = {1, 4, 9, 16, 25, 36, 49, 64, 81, 100}
\end{verbatim}
It is a natural question to ask 
what happens if $r$ is a binary relation and we try to evaluate the expression 
$r(x)$ but the set \texttt{\{ y : $\hspace*{-0.2cm}$[x, y] in r \}}
is either empty or contains
more than one element.
The program shown in figure \ref{fig:buggy-map.stlx} on page
\pageref{fig:buggy-map.stlx} answers this question.

\begin{figure}[!ht]
\centering
\begin{Verbatim}[ frame         = lines, 
                  framesep      = 0.3cm, 
                  firstnumber   = 1,
                  labelposition = bottomline,
                  numbers       = left,
                  numbersep     = -0.2cm,
                  xleftmargin   = 0.8cm,
                  xrightmargin  = 0.8cm,
                ]
    r := { [1, 1], [1, 4], [3, 3] };
    print( "r[1] = ", r[1] );
    print( "r[2] = ", r[2] );
    print( "{ r[1], r[2] } = ", { r[1], r[2] } );
    print( "r{1} = ", r{1} );
    print( "r{2} = ", r{2} );
\end{Verbatim}
\vspace*{-0.3cm}
\caption{A binary relation that is not a map.}
\label{fig:buggy-map.stlx}
\end{figure}


If the set \texttt{\{ y :[x, y] in r \}} is either empty or has more than one
element, then the expression \texttt{r[x]} is undefined.  In mathematics, an undefined value is
sometimes denoted as $\Omega$.  In \textsc{SetlX},
the undefined value is printed as ``\texttt{om}''.
If we try to add the undefined value to a set $m$, then $m$ is not changed.
Therefore, line 5 of the program just prints the empty set, as both \texttt{r[1]} and \texttt{r[2]}
are undefined.

We can use the notation  \texttt{r\{x\}} instead of \texttt{r[x]} to avoid undefined values.
For a binary relation $r$ and an object $x$, the expression $r\{x\}$ is defined as
follows: 
\\[0.2cm]
\hspace*{1.3cm} 
\texttt{r\{x\} := \{ y :[x, y] in r \}}.
\\[0.2cm]
Therefore, the  program from Figure \ref{fig:buggy-map.stlx} yields the following results:
\begin{verbatim}
    r[1] = om
    r[2] = om
    { r[1], r[2] } = {}
    r{1} = {1, 4}
    r{2} = {}
\end{verbatim}

\noindent
\textbf{Remark}:
This section has shown that binary relations can be used to represent the data type of a
\emph{dictionary}.  Other programming languages, for example \textsl{Perl} \cite{Wall92},
provide \emph{associative arrays} to represent dictionaries.  In most of the scripting languages
providing associative arrays, these associative arrays are implemented as 
\href{http://en.wikipedia.org/wiki/Hash_table}{\emph{hash tables}}.  In the programming language
\textsc{Setl}, sets and relations were also represented via hash tables.
In contrast, the sets (and therefore the dictionaries) in \textsc{SetlX} are implemented as 
\href{http://en.wikipedia.org/wiki/Red-black_tree}{\emph{red-black trees}}.
Although implementing sets as red-black trees is slightly slower than a hash table based
implementation, the advantage of using red-black trees is that they support a number of operations
that are not available when using hash-tables.  The efficient implementation of the functions \texttt{first}
and \texttt{last} would be impossible if sets had been represented as hash tables.


\section{Procedures}
Although functions can be represented as binary relations, this is not the preferred way
to represent functions.  After all, using a relation to represent a function
has a big memory footprint and also requires to compute all possible function
values regardless of their later use.  Therefore, the preferred way to code a function is
to use a \emph{procedure}.  For example, Figure \ref{fig:primes.stlx} defines a procedure
to compute all 
prime numbers up to a given natural number $n$.  The idea is to take the set $s$ of all numbers in
the range from $2$ upto $n$ and then to subtract the set of all non-trivial
products from this set.  This will leave us with the set of all prime numbers less or
equal to $n$ as a natural number is prime if and only if it is not a non-trivial product.

\begin{figure}[!ht]
\centering
\begin{Verbatim}[ frame         = lines, 
                  framesep      = 0.3cm, 
                  firstnumber   = 1,
                  labelposition = bottomline,
                  numbers       = left,
                  numbersep     = -0.2cm,
                  xleftmargin   = 0.8cm,
                  xrightmargin  = 0.8cm,
                ]
    primes := procedure(n) { 
        s := { 2..n }; 
        return s - { p*q : [p, q] in s >< s }; 
    };
\end{Verbatim}
\vspace*{-0.3cm}
\caption{A procedure to compute the prime numbers.}
\label{fig:primes.stlx}
\end{figure}

In Figure \ref{fig:primes.stlx}, the block starting with ``\texttt{procedure \{}'' and
ending with the closing brace ``\texttt{\}}'' defines a function.  This function is then
assigned to the variable \texttt{primes}.  Therefore, a function is just another kind of a
value.  Conceptually, the type of functions is not different from the type of sets or
strings:  A function can be assigned to a variable, it can be used as an argument to
another function and it can also be returned from another function.  To summarize,
functions are \href{http://en.wikipedia.org/wiki/First-class_function}{\emph{first class citizens}} 
in \setlx.  The ramifications of
this fact will be explained in Chapter \ref{chapter:closures} on functional programming
and \emph{closures}.

\section{Strings}
Any sequence of characters enclosed in double quotes is considered a string.
Strings can be concatenated using the infix operator ``\texttt{+}'', so
\\[0.2cm]
\hspace*{1.3cm}
\texttt{\symbol{34}abc\symbol{34} + \symbol{34}xyz\symbol{34}}
\\[0.2cm]
yields the string \texttt{\symbol{34}abcxyz\symbol{34}} as a result.
In order to concatenate multiple instances of the same string, we can use the infix
operator ``\texttt{*}''.  For a string $s$ and a natural number $n$, the expression
\\[0.2cm]
\hspace*{1.3cm}
\texttt{$s$ * $n$}
\\[0.2cm]
returns a string that consists of $n$ copies of the string $s$.  For example, the
expression 
\\[0.2cm]
\hspace*{1.3cm}
\texttt{"abc" * 3}
\\[0.2cm]
yields the result
\\[0.2cm]
\hspace*{1.3cm}
\texttt{"abcabcabc"}.
\\[0.2cm]
This also works when the position of the string and the number are exchanged.  Therefore,
\\[0.2cm]
\hspace*{1.3cm}
\texttt{3 * "abc"}
\\[0.2cm]
also yields \texttt{"abcabcabc"}.


In order to extract the $i$-th character of the string $s$, we can use the expression
\\[0.2cm]
\hspace*{1.3cm}
\texttt{$s$[$i$]}.
\\[0.2cm]
The slicing operators work similar to lists, for example, if $s$ has the value 
\texttt{\symbol{34}abcdef\symbol{34}}, then 
\\[0.2cm]
\hspace*{1.3cm}
\texttt{s[2..5];}
\\[0.2cm]
yields the result \texttt{\symbol{34}bcde\symbol{34}}, while 
\\[0.2cm]
\hspace*{1.3cm}
\texttt{s[2..];}
\\[0.2cm]
yields \texttt{\symbol{34}bcdef\symbol{34}} and 
\\[0.2cm]
\hspace*{1.3cm}
\texttt{s[..5];}
\\[0.2cm]
gives \texttt{\symbol{34}abcde\symbol{34}}.  

\setlx\ provides \emph{string interpolation}:
If a string contains a substring enclosed in ``\texttt{\symbol{36}}''-symbols, then \setlx\ 
parses this substring as an expression, evaluates this expression, and then substitutes the result
into the string.  For example, if the variable \texttt{n} has the value $6$, the command
\\[0.2cm]
\hspace*{1.3cm}
\texttt{print(\symbol{34}\symbol{36}n\symbol{36}! = \symbol{36}n!\symbol{36}\symbol{34})};
\\[0.2cm]
will print
\\[0.2cm]
\hspace*{1.3cm}
\texttt{6! = 720}.
\\[0.2cm]
In order to insert a literal ``\texttt{\symbol{36}}''-symbol into a string,  the
``\texttt{\symbol{36}}''-symbol has to be escaped with a backslash.  For example, the command
\\[0.2cm]
\hspace*{1.3cm}
\texttt{print(\symbol{34}a single \symbol{92}\symbol{36}-symbol\symbol{34});}
\\[0.2cm]
prints the text
\\[0.2cm]
\hspace*{1.3cm}
\texttt{a single \symbol{36}-symbol}.
\\[0.2cm]
String interpolation is turned off by the prefix operator ``\texttt{@}'':
The command
\\[0.2cm]
\hspace*{1.3cm}
\texttt{print(@\symbol{34}\symbol{36}n\symbol{36}! = \symbol{36}n!\symbol{36}\symbol{34});}
\\[0.2cm]
prints
\\[0.2cm]
\hspace*{1.3cm}
\texttt{\symbol{36}n\symbol{36}! = \symbol{36}n!\symbol{36}}.

\subsection{Literal Strings}
Sometimes it is necessary to turn off any kind of preprocessing when using a string.  This
is achieved by enclosing the content of the string in single quotes.  These strings are
known as \emph{literal} strings.  For example, after
the assignment
\\[0.2cm]
\hspace*{1.3cm}
\texttt{s := '\symbol{92}n';}
\\[0.2cm]
the string \texttt{s} contains exactly two characters:  The first character is the
backslash ``\texttt{\symbol{92}}'', while the second character is the character ``\texttt{n}''.
If  instead we write
\\[0.2cm]
\hspace*{1.3cm}
\texttt{s := "\symbol{92}n";}
\\[0.2cm]
then the string \texttt{s} contains just one character, which is the newline character.
\textsc{SetlX} supports the same escape sequences as the language \texttt{C}.

As we have just seen, there is no replacement of escape sequences in a literal string.
There is no string interpolation either.  Therefore, the statement
\\[0.2cm]
\hspace*{1.3cm}
\texttt{print('\symbol{36}1+2\symbol{36}');}
\\[0.2cm]
prints the string ``\texttt{\symbol{36}1+2\symbol{36}}'', where we have added the opening
and closing quotes to delimit the string, they are not part of the string.
Later, literal strings will come in very handy when dealing with \emph{regular expressions}.

\noindent
\textbf{Remark}:
It should be noted that \textsc{SetlX} does not have a special data type to support single characters.
The reason is that in \textsc{SetlX} single characters are represented as strings of length one. 

\section{Terms}
\setlx\ provides 
\href{http://en.wikipedia.org/wiki/Term_(first-order_logic)#Terms}{\emph{first order terms}}
similar to the terms available in the programming language 
\href{http://en.wikipedia.org/wiki/Prolog}{\emph{Prolog}} \cite{sterling:94}.
Terms are built from \emph{functors} and \emph{arguments}.  To distinguish functors from function
symbols, a functor will start with a capital letter.  For example, the expression
\\[0.2cm]
\hspace*{1.3cm}
\texttt{F(1,\symbol{34}x\symbol{34})}
\\[0.2cm]
is a term with functor \texttt{F} and two arguments.  Here, the functor is just a name, it is 
\underline{not} a function that can be evaluated.  To demonstrate the usefulness of terms, consider
implementing \href{http://en.wikipedia.org/wiki/Binary_search_tree}{\emph{ordered binary trees}} in \setlx. 
There are two types of ordered binary tress:
\begin{enumerate}
\item The empty tree represents the empty set.  We use the functor \texttt{Nil} to represent an
      empty binary tree, so the term
      \\[0.2cm]
      \hspace*{1.3cm}
      $\mathtt{Nil()}$
      \\[0.2cm]
      codes the empty tree.
\item A non-empty binary tree has three components:
      \begin{enumerate}
      \item The \emph{root} node,
      \item the left subtree, and
      \item the right subtree.
      \end{enumerate}
      The root node stores one element $k$ and all elements in the left subtree $l$ have to be less
      than $k$, while all elements in the right subtree are bigger than $k$.  Therefore, a non-empty
      binary tree can be represented as the term
      \\[0.2cm]
      \hspace*{1.3cm}
      $\texttt{Node}(k, l, r)$,
      \\[0.2cm]
      where $k$ is the element stored at the root, $l$ is the left subtree and $r$ is the right subtree.
\end{enumerate}
For example, the term
\\[0.2cm]
\hspace*{1.3cm}
\texttt{Node(2,Node(1,Nil(),Nil), Node(3,Nil(), Nil()))}
\\[0.2cm]
is a typical binary tree.  At the root, this tree stores the element 2, the left subtree
stores the element 1 and the right subtree stores the element 3.

There are two functions to decompose a term and there is one function that constructs a term.
\begin{enumerate}
\item If $t$ is a term, then the expression
      \\[0.2cm]
      \hspace*{1.3cm}
      $\texttt{fct}(t)$
      \\[0.2cm]
      returns the functor of the term t as a string.  For example, the expression
      \\[0.2cm]
      \hspace*{1.3cm}
      \texttt{fct(Node(3,Nil(),Nil()))}
      \\[0.2cm]
      yields the result \texttt{\symbol{34}Node\symbol{34}}.
\item If $t$ is a term, then the expression
      \\[0.2cm]
      \hspace*{1.3cm}
      $\mathtt{args}(t)$
      \\[0.2cm]
      returns the list of arguments of the term $t$.  For example, the expression
      \\[0.2cm]
      \hspace*{1.3cm}
      \texttt{args(Node(3,Nil(),Nil()))}
      \\[0.2cm]
      yields the result 
      \\[0.2cm]
      \hspace*{1.3cm}
      \texttt{[3, Nil(), Nil()]}.
\item The function $\texttt{makeTerm}(f,l)$ constructs a term $t$ such that
      \\[0.2cm]
      \hspace*{1.3cm}
      $\mathtt{fct}(t) = f$  \quad and \quad $\mathtt{args}(t) = l$
      \\[0.2cm]
      holds.  For example,
      \\[0.2cm]
      \hspace*{1.3cm}
      \texttt{makeTerm(\symbol{34}Node\symbol{34}, 
             [ makeTerm(\symbol{34}Nil\symbol{34},[]), makeTerm(\symbol{34}Nil\symbol{34},[]) ])}
      \\[0.2cm]
      constructs the term
      \\[0.2cm]
      \hspace*{1.3cm}
      \texttt{Node(3,Nil(),Nil())}.
      \\[0.2cm]
      Of course,  this term can also be given directly as an expression.  For example, the
      statement 
      \\[0.2cm]
      \hspace*{1.3cm}
      \texttt{a := Node(3,Nil(),Nil());}
      \\[0.2cm]
      assigns the term \texttt{Node(3,Nil(),Nil())} to the variable \texttt{a}.
\end{enumerate}

\begin{figure}[!ht]
\centering
\begin{Verbatim}[ frame         = lines, 
                  framesep      = 0.3cm, 
                  firstnumber   = 1,
                  labelposition = bottomline,
                  numbers       = left,
                  numbersep     = -0.2cm,
                  xleftmargin   = 0.8cm,
                  xrightmargin  = 0.8cm,
                ]
    insert := procedure(m, k1) {
        switch {
            case fct(m) == "Nil" : 
                 return Node(k1, Nil(), Nil());
            case fct(m) == "Node": 
                 [ k2, l, r ] := args(m);
                 if (k1 == k2) {
                     return Node(k1, l, r);
                 } else if (compare(k1, k2) < 0) { 
                     return Node(k2, insert(l, k1), r);
                 } else {
                     return Node(k2, l, insert(r, k1));
                 }
        }
    };
\end{Verbatim}
\vspace*{-0.3cm}
\caption{Inserting an element into a binary tree.}
\label{fig:binary-tree-no-matching.stlx}
\end{figure}

Figure \ref{fig:binary-tree-no-matching.stlx} shows how terms can be used to implement
binary trees.  In this example, we define a function with the name \texttt{insert}.  This function
takes two arguments.  The first argument \texttt{m} is supposed to be a term representing an ordered binary
tree.  The second argument \texttt{k1} denotes the element that is to be inserted into the binary
tree \texttt{m}.   The implementation needs to distinguish two cases:
\begin{enumerate}
\item If the binary tree \texttt{m} is empty, then the function returns the tree
      \\[0.2cm]
      \hspace*{1.3cm}
      \texttt{Node(k1, Nil(), Nil())}.
      \\[0.2cm]
      This is a binary tree containing the number \texttt{k1} at its root, while both subtrees are
      empty.  In order to check whether \texttt{m} is indeed empty we use the functor of the term \texttt{m}.  We test in line
      3 whether this functor is \texttt{\symbol{34}Nil\symbol{34}}.
\item If the binary tree \texttt{m} is nonempty, the functor of \texttt{m} is
      \texttt{\symbol{34}Node\symbol{34}}.  In this case, we need to extract the arguments of this
      functor, which is done in line 6: The first argument \texttt{k2} is the element stored at the
      root, while the arguments \texttt{l} and \texttt{r} correspond to the left and the right
      subtree of \texttt{m} respectively.  The predefined function \texttt{compare} is used to
      compare the element \texttt{k1} that is to be inserted into the tree with the element
      \texttt{k2}, which is the element at the root of the tree.

      The expression $\texttt{compare}(k1, k2)$ returns $-1$ if \texttt{k1} is less than
      \texttt{k2}, it returns $+1$ is \texttt{k1} is greater than \texttt{k2}, and it returns $0$
      if \texttt{k1} and \texttt{k2} are the same.  The function \texttt{compare} is implemented
      all data types. 
\end{enumerate}
This example uses a number of features of \setlx\ that have not been introduced. The discussion of
the \texttt{switch} construct will be given in the next chapter.

Sometimes, the fact that a functor has to be written in uppercase is inconvenient.  For these cases,
there is an escape mechanism:  If the functor of a term is preceded be the symbol
``\texttt{\symbol{64}}'', then it may start with a lowercase letter.  For example, if we want to
have terms representing algebraic expressions involving trancendental functions, we could use the
same function names that are also used in \setlx\ to denote the corresponding trancendental
functions.  For example, we can write
\\[0.2cm]
\hspace*{1.3cm}
\texttt{t := \symbol{64}exp(\symbol{64}log(\symbol{64}x));}
\\[0.2cm]
to represent the term
\\[0.2cm]
\hspace*{1.3cm}
\texttt{exp(log(x))}.
\\[0.2cm]
The present discussion of terms is not complete and will be continued in the next chapter when we
discuss matching.


%%% local Variables: 
%%% mode: latex
%%% TeX-master: "tutorial.tex"
%%% End: 

