\appendix
\chapter{Graphical Library Functions}
The following appendix lists the name of the functions that support the
animation of algorithms in \textsc{SetlX}.  The functions are listed in
alphabetical order.  It should be noted that the graphical library is an experimental feature that
might be subject to considerable changes in the future.  It should also be noted that this library
has been derived by Markus Jagiella from the file
\href{http://introcs.cs.princeton.edu/java/stdlib/StdDraw.java.html}{StdDraw.java}
that is part of the 
\href{http://algs4.cs.princeton.edu/home/}{booksite}
accompanying the excellent book on algorithms written by Robert Sedgewick and Kevin Wayne
\cite{sedgewick:11}.   
\begin{enumerate}
\item \texttt{gfx\_addPlayPauseButton}

      \texttt{gfx\_addPlayPauseButton()}

      \texttt{gfx\_addPlayPauseButton(boolean add)}

      Sets the Play/Pause Button on the animation frame visible (no
      parameter or true) or invisible (false).   Default is
      invisible. 

      Parameters:

      \texttt{add} -- the flag to set the visibility
\item \texttt{gfx\_addSpeedSlider}

      \texttt{gfx\_addSpeedSlider()}

      \texttt{gfx\_addSpeedSlider(boolean add)}

      Creates a speed slider on the animation frame if called with either no 
      parameter or arguments \texttt{true}.
      Set the slider to invisible if the argument is \texttt{false}. 

      Parameters:

      \texttt{add} -- the flag to set the visibility
\item \texttt{gfx\_arc}

      \texttt{gfx\_arc(double x, double y, double r, double phi1, double phi2)}

      Draws an arc of radius \texttt{r}, centered on the point $\langle x, y \rangle$, 
      from angle \texttt{phi1} to \texttt{phi2}. The angles are given in degrees.

      Parameters:
      \begin{description}
        \item[\texttt{x}] -- the x coordinate of the center of the circle
        \item[\texttt{y}] -- the y coordinate of the center of the circle
        \item[\texttt{r}] -- the radius of the circle
        \item[\texttt{phi1}] -- the starting angle. 0 would mean an arc beginning at 3 o'clock.
        \item[\texttt{phi2}] -- the angle at the end of the arc.  

              For example, in order to get a 90 degree arc, \texttt{phi2} should be
              \texttt{phi1} + 90. 
      \end{description}
\item \texttt{gfx\_circle}

      \texttt{gfx\_circle(double x, double y, double r)}

      Draws a circle of radius \texttt{r}, centered on the point $\langle x, y \rangle$.
      The circle degenerates to a pixel if the radius is too small.

      Parameters: 

      \texttt{x} -- the $x$-coordinate of the center of the circle 

      \texttt{y} -- the $y$-coordinate of the center of the circle 

      \texttt{r} -- the radius of the circle
\item \texttt{gfx\_clear}

       \texttt{gfx\_clear()}

       \texttt{gfx\_clear(String color)}

       Clear the screen with the given color or the default color (white). Calls \texttt{gfx\_show()}.

       Parameters: 
       
       \texttt{color} -- the color of the background
\item \texttt{gfx\_ellipse}

      \texttt{gfx\_ellipse(double x, double y, double semiMajorAxis, double semiMinorAxis)}

      Draws an ellipse with given semimajor and semiminor axis centered on $\langle x,y \rangle$.

      Parameters: 

      \texttt{x} -- the $x$-coordinate of the center of the ellipse

      \texttt{y} -- the $y$ coordinate of the center of the ellipse 

      \texttt{semiMajorAxis} -- the length of the semimajor axis of the ellipse 

      \texttt{semiMinorAxis} -- the length of the semiminor axis of the ellipse
\item \texttt{gfx\_filledCircle}

      \texttt{gfx\_filledCircle(double x, double y, double r)}

      Draws a filled circle of radius \texttt{r}, centered on $\langle x, y \rangle$.  The circle
      degenerates to a pixel if the radius is too small.

      Parameters:

      \texttt{x} -- the $x$-coordinate of the center of the circle

      \texttt{y} -- the $y$-coordinate of the center of the circle 

      \texttt{r} -- the radius of the circle

\item \texttt{gfx\_filledEllipse}

      \texttt{gfx\_filledEllipse(double x, double y, double semiMajorAxis, double semiMinorAxis)}

      Draws an ellipse with given semimajor and semiminor axis centered on $\langle x,y \rangle$.

      Parameters: 

      \texttt{x} -- the $x$-coordinate of the center of the ellipse

      \texttt{y} -- the $y$-coordinate of the center of the ellipse 

      \texttt{semiMajorAxis} -- the length of the semimajor axis of the ellipse 

      \texttt{semiMinorAxis} -- the length of the semiminor axis of the ellipse
\item \texttt{gfx\_filledPolygon}

      \texttt{gfx\_filledPolygon(double[] x, double[] y)}

      Draws a filled polygon with the given $\langle x[i], y[i] \rangle$ coordinates.

      Parameters:

      \texttt{x} -- an array of all the $x$-coordinates of the polygon 

      \texttt{y} -- an array of all the $y$-coordinates of the polygon
\item \texttt{gfx\_filledSquare}

      \texttt{gfx\_filledSquare(double x, double y, double r)}

      Draws a filled square of side length $2 \cdot r$, centered on $\langle x, y \rangle$.
      This square degenerates to a pixel if $r$ is too small.

      Parameters:

      \texttt{x} - the $x$-coordinate of the center of the square 

      \texttt{y} - the $y$ coordinate of the center of the square 

      \texttt{r} - the radius $r$ is half the length of any side of the square
\item \texttt{gfx\_getFont}

      \texttt{String gfx\_getFont()}

      Returns the name of the current font i.e. ``\texttt{Arial}''.
\item \texttt{gfx\_getPenColor}

      \texttt{String gfx\_getPenColor()}

      Returns the current color of the pen i.e.: ``\texttt{BLACK}''.
\item \texttt{gfx\_getPenRadius}

      \texttt{double gfx\_getPenRadius()}

      Returns the current pen radius.
\item \texttt{gfx\_hasNextKeyTyped}

      \texttt{boolean gfx\_hasNextKeyTyped()}

      Returns \texttt{true} if the user has typed a key, \texttt{false} otherwise.
\item \texttt{gfx\_isKeyPressed}

      \texttt{gfx\_isKeyPressed(int keyCode)}

      Returns \texttt{true} if the key with the given \texttt{keycode} is currently pressed and
      \texttt{false} otherwise. 

      Parameters:

      \texttt{keyCode} -- the physical key code of the Java programming language
\item \texttt{gfx\_isPaused}

      \texttt{boolean gfx\_isPaused()}

      Returns \texttt{true} if the animation is paused when this statement is executed and
      \texttt{false} otherwise.
\item \texttt{gfx\_line}

      \texttt{gfx\_line(double x0, double y0, double x1, double y1)}

      Draws a line from the point $\langle x_0, y_0 \rangle$ to the point $\langle x_1, y_1 \rangle$.

      Parameters:

      \texttt{x0} -- the $x$-coordinate of the starting point 

      \texttt{y0} -- the $y$-coordinate of the starting point 

      \texttt{x1} -- the $x$-coordinate of the destination point 

      \texttt{y1} -- the $y$ coordinate of the destination point
\item \texttt{gfx\_mouseX}

      \texttt{double gfx\_mouseX()}

      Returns the value of the $x$-coordinate of the mouse.
\item \texttt{gfx\_mouseY}

      \texttt{double gfx\_mouseY()}

      Returns the value of the $y$-coordinate of the mouse.
\item \texttt{gfx\_nextKeyTyped}

      \texttt{String gfx\_nextKeyTyped()}

      Returns the next unicode character that was typed by the user. This function cannot
      identify action keys such as ``F1'' or arrow keys.
\item \texttt{gfx\_picture}

      \texttt{gfx\_picture(double x, double y, String s)}

      \texttt{gfx\_picture(double x, double y, String s, double w, double h)}

      Draws a picture in \texttt{gif}, \texttt{jpg}, or \texttt{png} format centered 
      on the point $\langle x, y \rangle$.  If the parameters \texttt{w} and \texttt{h}
      are given, the picture is rescaled to a width of \texttt{w} and a height of $h$. 
      If \texttt{w} or \texttt{h} is too small, the picture degenerates into a pixel. 
      This function calls \texttt{gfx\_show()}.

      Parameters:

      \texttt{x} -- the $x$-coordinate of the  center  of the image 

      \texttt{y} -- the $y$-coordinate of the  center  of the image 

      \texttt{s} -- the name of the file containing the image, i.e. ``\texttt{flower.gif}'' 

      \texttt{w} -- the width of the image 
      
      \texttt{h} -- the height of the image
\item \texttt{gfx\_point}

      \texttt{gfx\_point(double x, double y)}

      Draws a point at the coordinates $\langle x, y \rangle$.
      
      Parameters:

      \texttt{x} - the $x$-coordinate of the point to be drawn

      \texttt{y} - the $y$-coordinate of the point to be drawn
\item \texttt{gfx\_polygon}

      \texttt{gfx\_polygon(double[] x, double[] y)}

      Draws a polygon with the given $\langle x[i], y[i]\rangle$ coordinates.

      Parameters:

      \texttt{x} - an array of all the $x$-coordinates of the polygon

      \texttt{y} - an array of all the $y$-coordinates of the polygon
\item \texttt{gfx\_rectangle}

      \texttt{gfx\_rectangle(double x, double y, double halfWidth, double halfHeight)}

      Draws a rectangle of given half width and half height centered on $\langle x,y \rangle$.

      Parameters:

      \texttt{x} -- the $x$-coordinate of the center of the rectangle

      \texttt{y} -- the $y$-coordinate of the center of the rectangle

      \texttt{halfWidth} -- the width of the rectangle is given as $2 \cdot \mathtt{halfWidth}$

      \texttt{halfHeight} -- the height of the rectangle is given as $2 \cdot \mathtt{halfHeight}$
\item \texttt{gfx\_setCanvasSize}
      \texttt{gfx\_setCanvasSize(int w, int h)}

      Sets the window size to \texttt{w}$\times$\texttt{h} pixels.

      Parameters:
      
      \texttt{w} -- the width as a number of pixels 

      \texttt{h} -- the height as a number of pixels
\item \texttt{gfx\_setPaused}

      \texttt{gfx\_setPaused(boolean paused)}

      Sets the start value of the \texttt{Play/Pause} button and determines whether the animation 
      is started in running or paused mode.

      Parameters:

      \texttt{paused} -- if this is true the animation is paused
\item \texttt{gfx\_setPenColor}

      \texttt{gfx\_setPenColor()}

      \texttt{gfx\_setPenColor(String color)}

      Sets the pen color to the given color or the default (\texttt{BLACK}) if no color is given. 
      The available pen colors are \texttt{BLACK}, 
      \texttt{BLUE},
      \texttt{CYAN},
      \texttt{DARKGRAY},
      \texttt{GRAY},
      \texttt{GREEN},
      \texttt{LIGHT\_GRAY},
      \texttt{MAGENTA},
      \texttt{ORANGE},
      \texttt{PINK},
      \texttt{RED},
      \texttt{WHITE}, and 
      \texttt{YELLOW}.

      Parameters:

      \texttt{color} -- the color of the pen
\item \texttt{gfx\_setPenColorRGB}

      \texttt{gfx\_setPenColorRGB(double r, double g, double b)}

      Sets the pen color to the color specified by the values of \texttt{r},
      \texttt{g}, and \texttt{b}.

      Parameters:

      \texttt{r} -- the luminosity of the red component of the color

      \texttt{g} -- the luminosity of the green component of the color

      \texttt{b} -- the luminosity of the blue component of the color

\item \texttt{gfx\_setPenRadius}

      \texttt{gfx\_setPenRadius()}

      \texttt{gfx\_setPenRadius(double r)}

      Sets the pen size to the given size or the default if no parameters are passed.

      Parameters:

      \texttt{r} -- the radius of the pen
\item \texttt{gfx\_setFont}

      \texttt{gfx\_setFont(String fontName)}

      \texttt{gfx\_setFont()}

      Changes the current font or restores the default font if no argument is passed.
      
      Parameters:

      \texttt{fontName} -- the name of the font i.e. ``Arial''
\item \texttt{gfx\_setScale}

      \texttt{gfx\_setScale(double min, double max)}

      Set the $x$ and $y$ scale to the given values or the default 
      if no parameters are passed (a border is added to the values). 
      Has the same effect as the calls  \texttt{setXscale(min,max)} and \texttt{setYscale(min,max)}
      combined.

      Parameters:

      \texttt{min} -- the minimum value of the $x$ and $y$ scale 

      \texttt{max} -- the maximum value of the X and Y scale
\item \texttt{gfx\_setXscale}

      \texttt{gfx\_setXscale()}

      \texttt{gfx\_setXscale(double min, double max)}

      Set the $x$ scale to the given values or the default if no parameters are passed (a border is
      added to the values). 

      Parameters:
      
      \texttt{min} -- the minimum value of the $x$ scale 

      \texttt{max} -- the maximum value of the $x$ scale
\item \texttt{gfx\_setYscale}

      \texttt{gfx\_setYscale()}

      \texttt{gfx\_setYscale(double min, double max)}

      Set the $y$ scale to the given values or the default if no parameters are passed (a border is
      added to the values). 

      Parameters:
      
      \texttt{min} -- the minimum value of the $y$ scale 

      \texttt{max} -- the maximum value of the $y$ scale
\item \texttt{gfx\_show}

      \texttt{gfx\_show()}

      Displays the animation on the screen.   
      Calling this method means that the screen will be redrawn after each 
      invocation of \texttt{line()}, \texttt{circle()}, or \texttt{square()}. This is the default.
\item \texttt{gfx\_show(int t)}
  
      Displays the animation on screen and pause for \texttt{t} milliseconds. Calling this method
      means that the screen will \underline{not} be redrawn after each invocation of 
      \texttt{line()}, \texttt{circle()}, or \texttt{square()}. This is
      useful when there are many invocation of these function necessary to draw a complete picture. 

      Parameters:

      \texttt{t} -- number of milliseconds to pause
\item \texttt{gfx\_square}

      \texttt{gfx\_square(double x, double y, double r)}

      Draws a square of side length $2 \cdot r$, centered on $\langle x, y \rangle$. 
      The square degenerates to a pixel if small $r$ is too small.

      Parameters:

      \texttt{x} -- the $x$-coordinate of the center of the square 

      \texttt{y} -- the $y$ coordinate of the center of the square 

      \texttt{r} -- the side length of the square is $2 \cdot r$
\item \texttt{gfx\_text}

      \texttt{gfx\_text(double x, double y, String s)}

      Write the given text string in the current font, centered on $\langle x, y \rangle$. 
      Calls \texttt{gfx\_show()}.
      
      Parameters:

      \texttt{x} -- the $x$ coordinate of the center of the text 

      \texttt{y} -- the $y$ coordinate of the center of the text 

      \texttt{s} -- the string
\item \texttt{gfx\_textLeft}

      \texttt{gfx\_textLeft(double x, double y, String s)}

      Write the given text string in the current font, right aligned at $\langle x, y \rangle$. 
      Calls \texttt{gfx\_show()}.

      Parameters:

      \texttt{x} -- the $x$ coordinate of the end of the text 

      \texttt{y} -- the $y$ coordinate of the end of the text 

      \texttt{s} -- the string
\item \texttt{gfx\_textRight}

      \texttt{gfx\_textRight(double x, double y, String s)}

      Write the given text string in the current font, left aligned at $\langle x, y \rangle$. 
      Calls \texttt{gfx\_show()}.

      Parameters:

      \texttt{x} -- the $x$ coordinate of the beginning of the text 

      \texttt{y} -- the $y$ coordinate of the beginning of the text 

      \texttt{s} -- the string
\end{enumerate}

%%% Local Variables: 
%%% mode: latex
%%% TeX-master: "tutorial"
%%% End: 
